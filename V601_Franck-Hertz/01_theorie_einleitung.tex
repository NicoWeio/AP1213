\section{Einleitung} % (fold)
\label{sec:einleitung}
	Der Franck-Hertz-Versuch hat eine gro"se Bedeutung in der fr"uhen Geschichte der Quantenmechanik.
	Er best"atigte bei seiner ersten Durchf"uhrung 1914 einige Annahmen der Bohrschen Postulate und erm"oglichte somit einen ersten Einblick in die innere Struktur der Atome.

	Durch die Anregung von Quecksilberatomen und die darauffolgenden Erscheinungen l"asst sich die Existenz von diskreten Energieniveaus best"atigen.
	Dies soll im Folgenden n"aher erl"autert werden.
	
\section{Theorie} % (fold)
\label{sec:theorie}
	\subsection{Wechselwirkungen in Atomh"ullen}
	\label{subsec:ww}
		Der Franck-Hertz-Versuch macht sich eine leicht veranschaulichbare Wechselwirkung zwischen freien Elektronen und an einen Kern gebundene H"ullenelektronen zu Nutzen.
		Ein freies Elektron kann mit einem Atom kollidieren, wobei je nach Energie des Elektrons ein elastischer oder ein unelastischer Sto"s auftritt.

		Das Elektron wird dabei vom Coulombpotential des H"ullenelektrons abgesto"sen.
		Falls das freie Elektron gen"ugend kinetische Energie tr"agt, kann es einen bestimmten Teil dieser Energie w"ahrend eines unelastischen Sto"ses an das H"ullenelektron abgeben.

		Das Atom befindet sich dann in einem angeregten, instabilen Zustand mit der Energie $E_1$, die gr"o"ser als die Energie $E_0$ des Grundzustandes ist.

		Nach dem Sto"s bewegt sich das freie Elektron auf Grund der geringeren Energie $E'$ mit einer verminderten Geschwindigkeit $v_2$ weiter.
		Zudem "andert es seine Richtung, in Abh"angigkeit zu der Geometrie des Sto"ses.
		Die "ubertragene Energie $\Delta E$ betr"agt

		\begin{equation}
			\Delta E = E_1 - E_0 = \frac{1}{2} m_0 \left(v_1^2 - v_2^2\right) \,.
		\end{equation}

		Die Energie $E'$ des freien Elektrons l"asst sich mit Hilfe der Gegenfeldmethode bestimmen, wodurch schlie"slich der Energieunterschied $\Delta E$ ermittelt werden kann.

	\subsection{Quecksilber als Sto"spartner}
	\label{subse:quecksilber}
		In diesem Versuch wird Quecksilber benutzt, da die Struktur der au"seren Schale (Hauptquantenzahl $n = 6$) eine leichte Anregung durch Elektronenst"o"se erlaubt.
		Das sto"sende Elektron vertauscht dabei mit dem H"ullenelektron.
		Das entspricht einem Umklappen des Elektronenspins, was eine Anregung in den ersten h"oheren Zustand zur Folge hat.

		Bei leichteren Kernen (z.B. Helium) ist dies praktisch unm"oglich.
		
	\clearpage