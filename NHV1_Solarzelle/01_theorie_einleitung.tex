\section{Einleitung}
	\label{sec:einleitung}
	In diesem Versuch soll der Wirkungsgrad einer Solarzelle bestimmt werden.
	Durch Messung von Strom und Spannung und mit Kenntniss der Intensit"at des einfallenden Lichtes l"asst sich dieser bestimmen.


\section{Theorie}
	\label{sec:theorie}

	\subsection{Aufbau und Funktion einer Solarzelle}
		\label{subsec:aufbau_funktion}
		Eine Solarzelle ist grunds"atzlich aufgebaut wie eine Diode.
		Eine p- und n-dotierte Halbleiterschicht werden zusammengebracht, wodurch am Grenzbereich ein starkes elektrisches Feld entsteht.

		Wird in diesem Bereich ein Photon absorbiert, das gen"ugend Energie besitzt um den Bandabstand des Atoms zu "uberschreiten, entsteht ein Elektron -- Loch Paar.
		Auf Grund des elektrischen Feldes, kann das Paar nicht sofort rekombinieren und Elektron und Loch werden abtransportiert.

		"Uber Kontakte an den Halbleiterschichten k"onnen die Ladungstr"ager dann zu einem Strom beitragen, der durch den Verbraucher abflie"st und dabei Arbeit verrichtet.

	\subsection{Technische Umsetzung}
		\label{sub:umsetzung}
		Als Rohstoff f"ur die meisten Solarzellen dient Silizium.
		Es kann g"unstig verarbeitet werden und der Umgang mit diesem Stoff ist gut erforscht.

		Die Zolarzellen werden auf verschiene Weisen hergestellt.
		Auf Grund der Kosteneffizienz hat sich kristallines Silizium durchgesetzt, ist im gro"sen Ma"se jedoch noch immer nicht konkurrenzf"ahig.
		Daher wird versucht, auch amorphes Silizium zu verwenden.

		

	\subsection{Berechnung des Wirkungsgrades $\eta$}
		\label{subsec:wirkungsgrad}
		F"ur den Wirkungsgrad $\eta$ der Zelle sind 