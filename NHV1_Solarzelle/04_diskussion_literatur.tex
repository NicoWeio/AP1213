\newpage
\section{Diskussion}
	\label{sec:diskussion}

	Alles in allem hat der Versuch die Schwierigkeiten bei der Solarzellennutzung hervorgehoben.
	Es wurde nicht nur durch die Messergebnisse deutlich, dass der Wirkungsgrad von der Strahlungsintensit"at und dem angelegten Widerstand abhg"angt, sondern auch stark von der Temperatur der Solarzelle.
	So ergab sich bei dem Abstand von $\SI{29}{\centi\meter}$ anf"anglich bei $\SI{5}{\ohm}$ f"ur $I_K = \SI{98.2}{\milli\ampere}$ und $U_0 = \SI{715}{\milli\volt}$.
	Nach einer erneuten Messung bei dem selben Widerstand ergab sich zu einem sp"ateren Zeitpunkt f"ur $I_K = \SI{99.8}{\milli\ampere}$ und $U_0 = \SI{813}{\milli\volt}$.\\ 
	Dies sind nicht vernachl"assigbare Differenzen, welche die Messergebnisse sehr leicht verf"alschen k"onnen.
	Dies ist aufgefallen, da sich bei der ersten Messung die Werte um das gedachte Maximum bei einer weiteren Feinmessung ver"andert hatten.
	Je kleiner der Abstand, um so gr"o"ser war der Effekt.\\
	Es ist daher sinnvoll bereits w"ahrend der Messung die Leistung zu berechnen und dann um das gefundene Maximum ein Feinmessung durchzuf"uhren, da die Temperatur zu diesem Zeitpunkt noch relativ gleich ist.\\
	Beim Auftragen von $U_0$ gegen $J$ konnte keine Aussage getroffen werden, da vier Messwerte daf"ur nicht ausreichend waren.\\
	Da die Lampe an einer nicht idealen Aufh"angung befestigt war, konnte nicht die optimale Intensit"at genutzt werden.
	Besonders beim verstellen der H"ohe ergaben sich Probleme beim Ausrichten der Lampe, sodass ein gleichbleibender Strahlenwinkel nicht gew"ahrleistet werden konnte.
	Schon kleine Drehungen bewirkten besonders bei kleinen Abst"anden gro"se Schwankungen in $I_K$.\\
	Trotzalledem stimmt der Literaturwert einer Silizium-Solarzelle  mit dem gemessenen Wert von ca. $16\%$ nahezu "uberein.
	Dies zeigt, dass sich trotz der m"oglichen Fehlerquellen ein gutes Ergebnis erzielen l"asst.

\begin{thebibliography}{9}
	\bibitem{artikel2} HAHN, Giso. \emph{Solarzellen aus Folien-Silizium}. Physik unserer Zeit 35, 2004/1: 20-27

	\bibitem{anleitung} Anleitung zum Versuch NHV1 Solarzellen

	\bibitem{wiki} Wikipedia. Halbleiter. http://de.wikipedia.org/wiki/Halbleiter. Stand 9. Dezember 2012
\end{thebibliography}