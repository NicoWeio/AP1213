\section{Diskussion}
	\label{diskussion}

	Auff"allig bei diesem Versuch ist die relativ gro"se Abweichung bei der Totzeitmessung des Z"ahlrohrs. F"ur die ermittelung der Totzeit mithilfe des Oszilloskops ergab sich $T_\mathrm{tot} = \SI{200}{\micro\second}$. Bei der Zwei-Quellen-Methode ergab sich $T_\mathrm{tot} = \SI{779.17 (30163)}{\micro\second}$.  
	Die Messung mithilfe des Oszilloskops ist jedoch auch sehr ungenau, da die Maxima nicht gut ablesbar sind und auch nur f"ur eine sehr kurze Zeit sichtbar sind.

	Die Probe war relativ schwach, sodass es schwierig war das Plateau des Z"ahlrohrs zu verlassen.

	Der Unterschied wurde erst bei der Zwei-Quellen-Methode deutlich, als mehr als die doppelte Messzeit bei gleichem Abstand benutzt werden musste, um an die n"otigen 10000 Ereignisse heranzukommen.

	

\begin{thebibliography}{9}
	\bibitem{anleitung} Physikalisches Anf"angerpraktikum der TU Dortmund: Versuch Nr. 703 - Das Geiger-M"uller-Z"ahlrohr. Stand: April 2013.
\end{thebibliography}
