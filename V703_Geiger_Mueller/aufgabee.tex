
\begin{table}[!h]
\begin{center}
\begin{tabular}{|r|r|r|r|r|r|r|r|r|}
\hline
U[$\SI{}{\volt}$] & N & $\Delta$N & N[$\SI{}{1\per\second}$] & $\Delta$N[$\SI{}{1\per\second}$] & I[$\SI{}{\micro\ampere}$] & $\Delta$I[$\SI{}{\micro\ampere}$] & Q $\backslash$T[$\SI{1 e10}{e}$] & $\Delta$Q$\backslash$T[$\SI{1 e10}{e}$] \\
\hline
\hline
480	 &   10725 & 104 & 89.38  & 0.87 &	0.4	& 0.1 & 2.79 & 0.6998\\
580	 &   11039 & 105 & 91.99  & 0.88 &	0.5	& 0.1 & 3.40 & 0.6802\\
600	 &   10734 & 104 & 89.45  & 0.87 &	0.6	& 0.1 & 4.19 & 0.6992\\
620	 &   10853 & 104 & 90.44  & 0.87 &	0.6	& 0.1 & 4.14 & 0.6916\\
640	 &   10788 & 104 & 90.44  & 0.87 &	0.6	& 0.1 & 4.17 & 0.6957\\
\hline
\end{tabular}
\caption[Aufgabe e]{Messdaten zur Messung der pro Teilchen vom Z"ahlrohr freigesetzten Ladungsmenge}
\label{tabellee}
\end{center}
\end{table}