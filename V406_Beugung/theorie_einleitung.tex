\section{Einleitung}
	\label{sec:einleitung}
	In diesem Versuch wird das Verhalten von monochromatischem Licht bei Beugung an d"unnen Spalten untersucht.\\
	Unter der Annahmen, dass sich Licht wellenartig ausbreitet wird der Intensitätsverlauf des Beugungsmusters hinter einem Einfach-, sowie einem Doppelspalt untersucht.
	Aus den Messwerten l"asst sich schlie"slich auf die Spaltbreite schlie"sen.

\section{Theorie}
	\label{sec:theorie}

	Im Folgenden werden verschiedenen Annahmen gemacht, die die Beschreibung des Lichtes vereinfachen.
	Dadurch k"onnen die Ph"anomene dieses Experimentes gut erkl"art werden.

	\subsection{Das Heuygenssche Prinzip}

		Um die Natur des Lichtes detailliert beschreiben zu k"onnen, muss man es quantenmechanisch betrachten.
		F"ur etliche Ph"anomene reicht es jedoch aus, "uber gro"se Zahlen von Lichtquanten zu mitteln und diese durch das klassische Wellenmodell n"aherungsweise zu beschreiben.\\

		Das Heugenssche Prinzip geht von der Welleneigenschaft des Lichtes aus.
		Es besagt, dass jeder Punkt einer Wellenfl"ache zur gleichen Zeit Elementarwellen aussendet.
		Diese Kugelwellen interferieren miteineander und bilden eine neue Wellenfront,
		die die Einh"ullende der Elementarwellen ist.
		Die "Uberlagerung aller Elementarwellen an einem Ort im Raum beschreibt dann den dortigen Schwingungszustand der Welle.

	\subsection{Fresnel- und Fraunhoferbeugung}
		\label{subsec:beugung}

		F"ur die Beschreibung von Beugungserscheinungen gibt es grunds"atzlich zwei verschiedene Versuchsanordnungen.

		\subsubsection{Fresnelbeugung}
			\label{subsubsec:fresnel}

			Die Fresnelsche Anordnung betrachtet eine Lichtquelle, die sich im endlichen Abstand vor dem Spalt befindet.
			Dadruch divergieren die Strahlenb"undel und das Licht wird am Spalt in unterschiedlichen Winkeln gebeugt.
			Schlu"sfolgerung auf den Versuchsaufbau durch Messung des Intensit"atsverlaufes werden damit sehr schwierig.

		\subsubsection{Fraunhoferbeugung}
			\label{subsubsec:fraunhofer}

			Diese Anordnung geht von parallelen Lichtb"undeln aus, die von einer unendlich weit entferten Lichtquelle entsandt werden.
			Hierdurch werden Lichtb"undel gleicher Phase im gleichen Winkel abelenkt.
			Die Beschreibung wird hier wesentlich einfacher, weil lediglich der Gangunterschied bei \emph{einem} Winkel, betrachtet werden muss.
			Die unendlich entfernte Lichtquelle l"asst sich gut durch einen Laser realisieren.\\

			Aus diesem Grund und weil der Fraunhoferaufbau eine leichtere Beschreibung liefert, wird der Versuch damit durchgef"urt.

	\subsection{Theoretisches Beugungsmuster}