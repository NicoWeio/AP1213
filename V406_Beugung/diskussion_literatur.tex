\section{Diskussion}
	\label{sec:diskussion}

	Allgemein l"asst sich sagen, dass sich die theoretisch hergeleiteten Beziehungen besonders gut in den ersten beiden Messreihen wiedergespiegelt haben.
	Die Auswertung dieser Daten lieferte Werte f"ur Spaltbreite und -abstand, die sehr nahe an den Herstellerangaben lagen.

	Die dritte Messreihe lieferte auf den ersten Blick "au"serst diffuse Werte. Dies lag an der geringen Aufl"osung der Messpunkte.
	Mit mehr Messpunkte h"atte ein Fit hier genauere Werte f"ur Spaltbreite- und abstand des Doppelspaltes liefern k"onnen. Dennoch waren die so gewonnenen Werte durchaus akzeptabel.

	Die Fehlerquellen waren bei allen Messungen jedoch sehr gro"s.
	Die zuvor gemessene Dunkelspannung $I_\mathrm{du}$ hat h"ochst empfindlich auf Ver"anderungen der Lichtverh"altnisse im Laborraum reagiert.
	Durch Ein- und Ausschalten der Lampen an anderen Versuchstischen oder das "offnen der Eingangst"ur konnten die Messwerte somit leicht beeintr"achtigt werden.
	Zudem stellte das Justieren der Versuchsapparatur eine gro"se Fehlerquelle dar.
	Der Laser konnte lediglich durch eine Stellschraube in H"ohe und Ausrichtung fixiert werden.
	So war es schwierig, die Photozelle exakt zu treffen.
	Ebenso konnte nicht ausgeschlossen werden, dass die Photozelle genau senkrecht zum Laser zeigte.
	Dadurch traf auf einer Seite eventuell mehr licht frontal auf die Zelle, als auf der anderen.
	Dies w"urde den Intensit"atsunterschied neben dem Hauptmaximum in unseren Messdaten erkl"aren.

	Schlie"slich l"asst sich "uber die Messung der Spaltbreite mit dem Mikroskop nur sagen, dass diese Methode "au"serst ungenau ist.
	Zwar war das Ergebnis in derselben Gr"osenordnung wie die Herstellerangabe, die Empfindlichkeit der Optik war jedoch so gro"s, dass wir die Aussagekraft dieser Messung stark bezweifeln.
	Die geringe Tiefensch"arfe des Mikroskopes machte eine genaue justierung auf die R"ander des Spaltes sehr schwierig.

\section{Literatur}

	Alle Grafiken wurden eigenst"andig mit Gnuplot oder pyplot erstellt oder aus der Versuchsanleitung \"Beugung am Spalt\" der TU Dortmund (Stand 29.10.12) entnommen.