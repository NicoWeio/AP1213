\clearpage
\section{Auswertung}

\subsection{Fehlerrechnung} % (fold)
\label{sub:fehlerrechnung}

Folgende Fehlerformeln werden in dieser Auswertung verwendet:

\begin{eqnarray}
	\Delta T'_\mathrm{1,2} &=& \left((|2t| \Delta a)^2 + (\Delta b)^2\right)^\frac{1}{2}\\
	\Delta W &=& \left(\frac{1}{n(n-1)}\sum_1^n (x_\mathrm{i}-\bar{x})^2\right)^\frac{1}{2}\\
	\Delta \nu &=& \left((|1/\bar{W}| \Delta T'_\mathrm{1})^2 + (|T'_\mathrm{1}/\bar{W}^2| \Delta W)^2\right)^\frac{1}{2}\\
	\Delta T_\mathrm{1,2} &=& \left((|t^2| \Delta a)^2 + (|t| \Delta b)^2 + (\Delta c)^2\right)^\frac{1}{2}\\
	\Delta \nu_\mathrm{theo} &=& \left(\left(|\frac{2T_\mathrm{1}-T_\mathrm{2}}{(T_\mathrm{1}-T_\mathrm{2})^2}| \Delta T_\mathrm{1}\right)^2 + \left(|\frac{T_\mathrm{1}}{(T_\mathrm{1}-T_\mathrm{2})^2}| \Delta T_\mathrm{2}\right)^2\right)^\frac{1}{2}\\
	\Delta \frac{\mathrm{d}m}{\mathrm{d}t} &=& \left(\left(|\frac{1}{L}|\Delta T'_\mathrm{2}\right)^2 + \left(|\frac{T'_\mathrm{2}}{L^2}| \Delta L \right)^2\right)^\frac{1}{2}	
\end{eqnarray}

\subsection{Approximierung der Temperaturkurven} % (fold)
\label{sub:approximieren_der_temperaturkurven}


\begin{table}[!h]
\begin{center}
\begin{tabular}{|r|r|r|r|r|r|}
\hline
t[$\SI{}{\minute}$] & $T_\mathrm{1}[\SI{}{\kelvin}$] & $T_\mathrm{2}[\SI{}{\kelvin}$] & $p_\mathrm{a}[\SI{}{\bar}]$ & $p_\mathrm{b}[\SI{}{\bar}]$ & A[$\SI{}{\watt}$]\\
\hline
\hline

 1 &	23,7 &	19,6 &	2,0 &	 6,50 &	200\\
 2 &	25,9 &	17,6 &	2,1 &	 7,00 &	200\\
 3 &	28,3 &	15,4 &	2,2 &	 7,25 &	205\\
 4 &	30,6 &	13,4 &	2,2 &	 7,75 &	210\\
 5 &	32,9 &	11,2 &	2,2 &	 8,25 &	210\\
 6 &	35,1 &	 9,4 &	2,2 &	 8,75 &	210\\
 7 &	37,3 &	 7,5 &	2,2 &	 9,00 &	215\\
 8 &	39,3 &	 5,7 &	2,2 &	 9,50 &	215\\
 9 &	41,2 &	 4,0 &	2,2 &	10,00 &	215\\
10 &	43,1 &	 2,5 &	2,2 &	10,50 &	215\\
11 &	44,9 &	 1,4 &	2,2 &	11,00 &	215\\
12 &	46,7 &	 0,7 &	2,2 &	11,50 &	215\\
13 &	48,3 &	 0,1 &	2,2 &	12,00 &	215\\
14 &	49,8 &	-0,3 &	2,2 &	12,50 &	215\\

\hline
\end{tabular}
\caption[]{Aufgenommene Messgr"o"sen zur Bestimmung der G"uteziffer der Apparatur, dem Massendurchsatz des Transportgases und der mechanischen Leistung des Kompressors.}
\label{daten}
\end{center}
\end{table}

\begin{figure}[!h]
	\centering
	\includegraphics[width = 10cm]{img/fit.pdf}
	\caption{Darstellung der Temperatur in Abh"angigkeit von der Zeit.}
	\label{fit}
\end{figure}

F"ur den fit kamen folgende Funktionen in Frage:

\begin{eqnarray}
T(t) &=& A t^2 + B t + C \label{T11}\\
T(t) &=& \frac{A}{1+B t^\alpha}\\
T(t) &=& \frac{A t^\alpha}{1 + B t^\alpha} + C
\end{eqnarray}

Es wurde sich f"ur die Funktion \eqref{T11} entschieden, da diese die geringste Abweichung von den Messpunkten zeigte.
In Tabelle \ref{daten} sind die gemessenen Gr"o"sen aufgelistet. In Graphik \ref{fit} ist die Temperatur in Abh"angigkeit von der Zeit dargestellt.
Mittels einer nicht linearen Ausgleichsgerade der Form $A t^2 + B t +c$ ergaben sich die Werte aus Tabelle \ref{ergebnisse} f"ur A,B und C.


\begin{table}[!h]
\begin{center}
\begin{tabular}{|r|r|r|r|r|}
\hline
Konst & $T_\mathrm{1}[\SI{}{\kelvin}]$ & $\Delta T_\mathrm{1}[\SI{}{\kelvin}]$ & $T_\mathrm{2}[\SI{}{\kelvin}]$ & $\Delta T_\mathrm{1}[\SI{}{\kelvin}]$\\
\hline
\hline

A[$\SI{}{\kelvin\per\second^2}$] &	-0.0000101 &	 0.0000004 &	0.000021 &	0.000002\\
B[$\SI{}{\kelvin\per\second}$]   &	    0.0429 &	    0.0004 &	  -0.046 &	   0.002\\
C[$\SI{}{\kelvin}$]              &	    294.14 & 	      0.08 &	  295.90 &	    0.32\\

\hline
\end{tabular}
\caption[]{Ergebnisse eines nicht-linearen Fits zur Darstellung der Temperaturkurven $T_\mathrm{1}$ und $T_\mathrm{2}$ mittels $A*t^2+B*t+C$.}
\label{ergebnisse}
\end{center}
\end{table}

\subsection{G"uteziffer} % (fold)
\label{sub:g_uteziffer}


\begin{table}[h!]
\begin{center}
\begin{tabular}{|r|r|r|r|r|r|r|r|r|}
\hline
Wert x[$\SI{}{\second}$] & $T'_\mathrm{1}[\SI{}{\kelvin}$] & $\Delta T'_\mathrm{1}[\SI{}{\kelvin}$] & $\mathrm{d}Q1/\mathrm{d}t[\SI{}{\joule\per\second}]$ & $\Delta\mathrm{d}Q1/\mathrm{d}t[\SI{}{\joule\per\second}]$ & $\nu$ & $\Delta\nu$ & $\nu_\mathrm{theo}$ & $\Delta\nu_\mathrm{theo}$\\
\hline
\hline
240 &	0,038 &	0,0004 &	502,51 &	5,86 &	2,38 &	0,03 & 17,09 &	0,58\\
360 &	0,035 &	0,0005 &	470,50 &	6,51 &	2,23 &	0,03 & 11,76 &	0,38\\
480 &	0,033 &	0,0006 &	438,49 &	7,32 &	2,08 &	0,04 &  9,26 &	0,31\\
600 &	0,031 &	0,0006 &	406,48 &	8,25 &	1,93 &	0,04 &  7,83 &	0,29\\
\hline
\end{tabular}
\caption[]{Genutzte Werte zur Bestimmung der G"uteziffer.}
\label{guete}
\end{center}
\end{table}

F"ur den Mittelwert der abgelesenen Arbeit $A$ errechnet sich:

\begin{equation}
	A = \SI{211.97 (150)}{\watt}
\end{equation}

$T'_\mathrm{1}$ stellt dabei die Ableitung von $T_\mathrm{1}$ nach $t$ dar.
F"ur die G"uteziffer ergaben sich die Werte, welche in Tabelle \ref{guete} aufgelistet sind.
Der Wert f"ur $\nu_\mathrm{theo}$ berechnet sich nach Gleichung \eqref{nu_theo}.
Der Wert f"ur $\nu$ wurde aus Gleichung \eqref{nu_real} berechnet.

\subsection{Massendurchsatz} % (fold)
\label{sub:massendurchsatz}

\begin{figure}[!h]
	\centering
	\includegraphics[width = 10cm]{img/masse.pdf}
	\caption{Graphik zur Bestimmung der Verdampfungsw"arme L aus einer linearen Ausgleichsgerade.}
	\label{L}
\end{figure}


\begin{table}[h!]
\begin{center}
\begin{tabular}{|r|r|r|r|r|}
\hline
Wert x[$\SI{}{\second}$] & $T'_\mathrm{2}[\SI{}{\kelvin}$] & $\Delta T'_\mathrm{2}[\SI{}{\kelvin}$]& $\mathrm{d}m/\mathrm{d}t[\SI{}{\gram\per\second}$] & $\Delta\mathrm{d}m/\mathrm{d}t[\SI{}{\gram\per\second}$]\\
\hline
\hline
240 &	-0,040 &	0,002 &	-1,95 &	0,17\\
360 &	-0,031 &	0,002 &	-1,67 &	0,17\\
480 &	-0,026 &	0,003 &	-1,40 &	0,17\\
600 &	-0,021 &	0,003 &	-1,13 &	0,18\\
\hline
\end{tabular}
\caption[]{Genutzte Werte zur Bestimmung des Massendurchsatzes.}
\label{masse}
\end{center}
\end{table}

Die Werte zur Bestimmung des Massendurchsatzes sind in Tabelle \ref{masse} dargestellt.
Die Verdampfungsw"arme L ist aus einer linearen Ausgleichsgeraden errechenbar. Nun l"asst sich durch umstellen der Gleichung \eqref{masse_theo} nach $\mathrm{d}m/\mathrm{d}t$ der Massendurchsatz berechnen.

Es gilt:

\begin{equation}
	ln(p) = \frac{L}{R}\frac{1}{T}+ B
\end{equation}

Dabei stellt $R \approx$ 8,31 die ideale Gaskonstante dar.

Die Ausgleichsrechnung mittels $f(x)=m x + b$ ergibt:

\begin{eqnarray}
	m = \SI{-29.3(18)}{}\\
	b = \SI{3.04(5)}{}
\end{eqnarray}

Daraus folgt:

\begin{equation}
	L = -m \cdot R = \SI{243.61 (1497)}{\joule\per\gram}
\end{equation}



\subsection{Mechanische Arbeit des Kompressors} % (fold)
\label{sub:mechanische_arbeit}


\begin{table}[!h]
\begin{center}
\begin{tabular}{|r|r|r|r|}
\hline
$\rho_\mathrm{0}[\SI{}{\gram\per\liter}$] & $\kappa$ & $p_\mathrm{a}[\SI{}{\bar}]$ & $\rho[\SI{}{\newton\per\meter^2}$]\\
\hline
\hline
5,51 & 1,14 & 2,2 & 3311,12\\
\hline
\end{tabular}
\caption[]{Werte zur Bestimmung der mechanischen Kompressorleistung.}
\label{arbeit1}
\end{center}
\end{table}

Die gegebenen Werte zur Bestimmung der mechanischen Arbeit des Kompressors sind in Tabelle \ref{arbeit1} dargestellt \cite{anleitung}. Der Wert f"ur $p_\mathrm{a}$ blieb "uber dem betrachteten Bereich konstant.

Mithilfe der Formel \ref{arbeitarbeit} kann nun die Arbeit errechnet werden. Die Ergebnisse sind in Tabelle \ref{arbeit2} dargestellt.


\begin{table}[!h]
\begin{center}
\begin{tabular}{|r|r|}
\hline
$p_\mathrm{b}[\SI{}{\bar}$] & N[$\SI{}{\watt}$]\\
\hline
\hline
 7,75 & 29,21\\
 8,75 & 25,47\\
 9,50 & 21,57\\
10,50 & 17,54\\
\hline
\end{tabular}
\caption[]{Berechnung der mechanische Arbeit des Kompressors.}
\label{arbeit2}
\end{center}
\end{table}

\clearpage