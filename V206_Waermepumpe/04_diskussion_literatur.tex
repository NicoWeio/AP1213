\section{Diskussion}
\label{sec:diskussion}

Der Versuch hat gut funktioniert. Die gemessenen G"uteziffern liegen weit von den theoretischen Werten entfernt, doch dies ist nicht ungew"ohnlich, da es zu erheblichen Verlusten bei den W"armetauschern und den anderen Apparaturteilen kommt. Zudem k"onnte es zu einer erw"armung des Kompressors kommen, wodurch keine adiabatische Kompression mehr gew"ahrleistet ist.
Bei der Auswahl der passensten Funktion wurde sich f"ur den $f(t)=At^2+Bt+C$ Fit entschieden, da dieser den kleinsten Fehler aufwies. Der Fit mit $f(t) = At^\alpha / (1+Bt^\alpha) + C$ ergab Fehler in der Gr"o"senordnung von $\SI{e14}{}$ und dieser w"are daher nicht aussagekr"aftig. Auch der $f(t) = A/(1+Bt^\alpha)$ Fit ergab Fehler in der Gr"o"senordnung $\SI{e13}{}$.

\begin{thebibliography}{9}
	\bibitem{anleitung} Physikalisches Anf"angerpraktikum der TU Dortmund: Versuch V206 - Die W"armepumpe. \url{http://129.217.224.2/HOMEPAGE/PHYSIKER/BACHELOR/AP/SKRIPT/V206.pdf}. Stand: Juli 2013.
\end{thebibliography}