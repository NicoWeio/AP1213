\section{Einleitung}
\label{sec:einleitung}
	Der Transport von W"armeenergie zwischen verschiedenen Reservoirs stellt eine essenzielle technische Anwendung physikalischer Mechanismen in unserem Alltag dar.
	Dabei lehrt die Erfahrung, dass W"armeenergie in einem abgeschlossenen System stets vom w"armeren in das k"uhlere Resrvoir "ubergeht.
	Durch Aufwendung "au"serer Arbeit l"asst sich dieser Prozess jedoch umkehren.

	\subsection{Ableitungen aus dem Energiesatz}
	\label{subsec:energiesatz}
		Der Energiesatz besagt, dass die Gesamtenergie eines abgeschlossenen Systemes zeitlich konstant bleibt.
		Das abgeschlossene System der W"armepumpe enth"alt unter idealisierter Betrachtung zwei W"armereservoirs, die W"arme "uber ein Transportmedium austauschen k"onnen.

		Wenn die "au"sere Arbeit $A$ am System verrichtet wird, besagt der Energiesatz f"ur die W"armemenge $Q_1$, die vom Transportmedium an das w"armere Reservoir abgegeben wird und f"ur die aus dem k"alteren Reservoir entnommene W"armemenge $Q_2$:
		\begin{equation*}
			Q_1 = Q_2 + A \,.
		\end{equation*}

		Damit wird der Quotient $\nu$
		\begin{equation*}
			\nu = \frac{Q_1}{A}
		\end{equation*}

		als G"uteziffer der W"armepumpe bezeichnet.
		Mit Hilfe des 2. Hauptsatzes der W"armelehre l"asst sich herleiten, dass
		\begin{equation*}
			\frac{Q_1}{T_1} - \frac{Q_2}{T_2} = 0
		\end{equation*}

		gilt.
		Hierbei bezeichnet $T_1$ die Temperatur des w"armeren und $T_2$ die Temperatur des k"alteren Reservoirs.
		Dies bezieht sich jedoch auf die idealisierten Umst"ande. In der Realit"at nimmt diese Gleichung auf Grund von Verlusten folgende Gestalt an:
		\begin{equation}
			\frac{Q_1}{T_1} - \frac{Q_2}{T_2} > 0 \,.
		\end{equation}

		Damit folgt f"ur die G"uteziffer $\nu_\mathrm{r}$ einer realen W"armepumpe:
		\begin{equation}
			\nu_\mathrm{r} < \frac{T_1}{T_1 - T_2}
		\end{equation}

		Eine gro"se G"uteziffer $\nu$ bedeutet dabei, dass nur eine geringe Arbeit $A$ geleistet werden muss, um eine gew"unschte W"armemenge $Q_2$ aus dem entsprechenden Reservoir zu entnehmen.
		Dabei kann die zugef"uhrte Arbeit $A$ viel kleiner sein als die bef"orderte Energie $Q_{1,\mathrm{rev}}$.
		Dies stellt den gro"sen Vorteil der W"armepumpe dar, denn bei mechanischen W"armegewinnungsverfahren ist die erlangte W"armemenge $Q_{1,\mathrm{direkt}}$ h"ochstens so gro"s wie die Arbeit $A$:
		\begin{eqnarray*}
			Q_{1,\mathrm{direkt}} & \leq & A \,, \\
			Q_{1,\mathrm{rev}} &  \leq & \frac{T_1}{T_1 - T_2} A \,.
		\end{eqnarray*}