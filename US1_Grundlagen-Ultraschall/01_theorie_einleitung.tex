\section{Einleitung} % (fold)
\label{sec:einleitung}
	Die Ultraschalltechnik findet eine weit verbreitete Anwendung.
	Sie wird h"aufig im medizinischen Bereich zur Diagnostik, sowie im technischen Bereich zur Materialanalyse benutzt.
	Die grundlegende Funktionsweise dieser Technik soll hier n"aher untersucht werden.
\section{Theorie} % (fold)
\label{sec:theorie}
	Schall bezeichnet im Allgemeinen Druckver"anderungen in einem Medium.
	In Luft und Fl"ussigkeiten breiten sich diese Druckver"anderungen als Longitudinalwellen aus.
	Die Frequenz $\nu$ dieser Wellen wird in vier Bereiche unterteilt.
	Der menschlichen H"orbereich liegt in etwa bei $\nu = \SI{16}{\hertz}$ bis $\nu = \SI{20}{\kilo \hertz}$.
	Bis zu einer Frequenz von etwa $\nu = \SI{1}{\giga \hertz}$ werden Schallwellen als Ultraschall bezeichnet.
	Bei noch h"oheren Frequenzen handelt es sich um Hyperschall.
	Der Bereich unter $\nu = \SI{16}{\hertz}$ wird als Infraschall bezeichnet.

	\subsection{Ausbreitung von Schallwellen}
	\label{subsec:ausbreitung}

		Eine eindimensionale Schallwelle $p$, die sich in $x$-Richtung ausbreitet wird beschrieben durch

		\begin{equation}
			p(x,t) = p_0 + v_0 Z \cos{(\omega t - kx)} \,.
		\end{equation}

		Dabei bezeichnet $p_0$ den Normaldruck, $v_0$ die Schallschnelle und $Z = \rho c$ die akustische Impedanz mit der Dichte $\rho$ des durchstrahlten Materials und der Schallgeschwindigkeit $c$.

		F"ur die Schallgeschwindigkeit $c_\mathrm{fl}$ in Fl"ussigkeiten und Gasen gilt der einfache Zusammenhang

		\begin{equation}
			c_\mathrm{fl} = \sqrt{\frac{1}{\kappa \rho}} \,,
		\end{equation}

		Mit der Kompressibilit"atszahl $\kappa$. \\

		In Festk"orpern ist h"angt die Schallgeschwindgkeit auf kompliziertere Art und Weise von der Beschaffenheit des Materials ab.
		Auf Grund der inneren Struktur und den daraus resultierenden Schubspannungen kann eine Schallwelle auch transversale Anteile enthalten.
		Die Schallgeschwindigkeit unterscheidet sich dann f"ur die transversale und die longitudinale Welle.
		F"ur den Longitudinalanteil der Welle gilt mit dem Elastizit"atsmodul $E$ des Festk"orpers

		\begin{equation}
			c_\mathrm{fe} = \sqrt{\frac{E}{\rho}} \,.
		\end{equation}

		
		\clearpage
		Auf dem Weg durch ein Material nimmt die Amplitude der Schallwelle exponentiell, abh"angig vom Absorptionskoeffizienten $\alpha$, ab:

		\begin{equation}
			I(x) \propto e^{\alpha x} \,.
		\end{equation}

		Wie bei elektromagnetischen Wellen werden Schallwellen an Grenzfl"achen ebenso reflektiert oder transmittiert.
		Bezeichnen $Z_1$ und $Z_2$ die akustischen Impedanzen der beiteiligten Materialien, gilt f"ur den Reflexionskoeffizienten $R$ und den Transmissionskoeffizienten $T$:

		\begin{eqnarray}
			R & = & \left(\frac{Z_1 - Z_2}{Z_1 + Z_2}\right)^2 \,, \\
			T & = & 1 - R \,.
		\end{eqnarray}

	\subsection{Erzeugung von Ultraschall}
	\label{subsec:erzeugung}
		F"ur die Erzeugung von Ultraschallwellen kann der reziproke piezze-elektrischen Effekt genutzt werden.
		Ein piezzo-elektrischer Kristall, wie zum Beispiel Quarz, besitzt die Eigenschaft, sich unter Einwirkung eines elektrischen Feldes zu verformen.
		In Umkehrung induziert dieser Kristall bei einer Krafteinwirkung ein elektrisches Feld.

		Wird ein solcher Kristall nun mit einem elektrischen Wechselfeld angeregt, emittiert er je nach Frequenz des Feldes Ultraschall.
		Bei Resonanz k"onnen damit gro"se Ultraschallamplituden erzeugt werden.

		Umgekehrt kann das Feld, welches der Kristall bei Ultraschalleinstrahlung erzeugt, gemessen werden, sodass er als Sender und Empf"anger dient. \\

	\subsection{Ultraschallverfahren}
	\label{subsec:verfahren}
		Zun"achst werden zwei Verfahren der Ultraschalltechnik vorgestellt.
		Beide Verfahren nutzen aus, dass Schall an St"orstellen in einem Medium (z.B. Verschmutzungen, Materialfehler) reflektiert und teilweise absorbiert wird.

		\begin{itemize}
			\item Beim \emph{Durchschallungsverfahren} wird die Probe zwischen eine Sender- und eine Empf"angersonde gespannt.
			Nachdem der Sender einen Schallimpuls erzeugt, trifft dieser Impuls nach einer gewissen Zeit $t$ auf den Empf"anger.
			Die Intensit"at des empfangenen Impulses sinkt mit der Gr"o"se der Probe.
			Zudem wird sie verkleinert, falls der Schallimpuls in der Probe auf eine St"orstelle trifft.

			\item Der Sender dient beim \emph{Impuls-Echo-Verfahren} auch als Empf"anger.
			Nachdem ein Schallimpuls gesendet wird, wird die Laufzeit $t$ und die Intensit"at des reflektierten Impulses gemessen.
			Mit Kenntnis der Schallgeschwindigkeit $c$ in der untersuchten Probe l"asst sich der Abstand $s$ der St"orstelle zur Sonde bestimmen:

			\begin{equation}
				s = \frac{1}{2}ct \,.
			\end{equation}
		\end{itemize}

		Die Laufzeitdiagramme, die durch die genannten Ultraschallverfahren aufgenommen werden, lassen sich wiederum auf verschiedene Weise auswerten.
		In der Medizin gibt es drei unterschiedliche Darstellungsarten:

		\begin{itemize}
			\item Mit dem Amplituden-Scan (\emph{A-Scan}) wird die Echoammplitude gegen die Laufzeit aufgetragen.

			\item Der Brightness-Scan (\emph{B-Scan}) stellt die Echoamplituden in Farbstufen dar.
			Durch Bewegung der Sonde kann dadurch ein zweidimensionales Bild erzeugt werden.

			\item Der Time-Motion-Scan (\emph{TM-Scan}) nimmt durch eine schnelle Abtastung der Probe, bzw. des zu untersuchenden K"orperteiles eine zeitliche Bildfolge auf, wodurch sogar Bewegungen innerhalb der Probe sichtbar gemacht werden k"onnen.
		\end{itemize}