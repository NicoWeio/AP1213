\section{Auswertung}
	\label{sec:auswertung}

	\subsection{Einseitige Einspannung}
	\label{subsec:einseitig}
		Zur Bestimmung des Elastizit"atsmodules der beiden Stangen, wird jeweils eine lineare Ausgleichsrechnung der Form $D(x) = a\chi$ mit Gleichung \eqref{eqn:einseitig} durchgef"uhrt.
		Hierf"ur wird die Durchbiegung $D$ gegen das Argument
		\begin{equation*}
			\chi = Lx^2 - \frac{x^3}{3}
		\end{equation*}

		aufgetragen. Aus der Steigung $a$ l"asst sich mit Kenntnis der Kraft $F$, die am Stab angreift und des Fl"achentr"agheitsmomentes $I$ die Elastizit"at $E$ bestimmen:
		\begin{eqnarray*}
			a & = & \frac{F}{2EI} \\
			\Leftrightarrow \quad E & = & \frac{F}{2aI} \,.
		\end{eqnarray*}

		F"ur die Kraft $F$ gilt mit der Erdbeschleunigung $g = \SI{9.81}{\newton \per \kilo \gram}$:
		\begin{equation*}
			F = mg
		\end{equation*}

		Zun"achst wird ein quadratischer Stab mit Breite $b$ gemessen. Hier gilt f"ur das Fl"achentr"agheitsmoment $I$ und den Fehler $\Delta I$:
		\begin{eqnarray*}
			I & = & \frac{b^4}{12} \,, \\
			\Delta I & = & \frac{b^3}{3}\Delta b \,.
		\end{eqnarray*}

		Die Breite des Stabes wird durch Mittelung "uber zehn Messwerte bestimmt und es folgt:
		\begin{eqnarray*}
			b & = & \SI{9.96(1)}{\milli \meter} \, \\
			\Rightarrow \quad I & = & \SI{820.1(27)}{\milli \meter \tothe{4}} \,. 
		\end{eqnarray*}

		Mit der Gewichtmasse $m$ folgt die Kraft $F$:
		\begin{eqnarray*}
			m & = & \SI{542.5}{\gram} \, \\
			\Rightarrow \quad F & = & \SI{5.322}{\newton} \,.
		\end{eqnarray*}

		Die lineare Ausgleichsrechnung liefert dann:
		\begin{eqnarray*}
			a & = & \SI{4.916(53)e-08}{\milli \meter \tothe{4}} \\
			\Rightarrow \quad E & = & \SI{65.947(743)}{\kilo \newton \per \milli \meter \cubed} \,.
		\end{eqnarray*}

