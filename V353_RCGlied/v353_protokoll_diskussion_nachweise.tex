\documentclass{scrartcl}
	%ngerman fur Umlaute, Anf"hrungszeichen, etc.%
	\usepackage{ngerman}

	%siunitx fur SI-Einheitesnsystem%
	\usepackage{siunitx}
	\sisetup{
	    locale=DE,
	    separate-uncertainty=true,
    	per-mode=fraction
	}

	%graphicx fur Bildeinbindung%
	\usepackage{graphicx}

	%Texteinzug vor Absatz entfernen%
	\parindent0pt

\begin{document}

	\section{Diskussion}

		Es l"asst sich sagen, dass dieser Versuch die Eigenschaften eines RC-Gliedes in einem Wechselstromkreis sehr gut demonstriert hat.
		Die Bestimmung der Zeitkonstanten $\lambda$ (Aufg. \ref{aufg_1}) lieferte zwar ein sch"ones Ergebnis,
		"uber die Aussagekraft dieses Ergebnisses lie"s sich jedoch nicht viel sagen,
		da wir keinen Vergleichswert hatten
		und systematische Fehler nicht ausschlie"sen konnten.
		Die Messaufgaben \ref{aufg_2} und\ref{aufg_3} verdeutlichten die in der Theorie hergeleiteten zusammenh"ange wiederum sehr gut.
		Schlie"slich lieferte der letzte Versuchsteil besonders eindrucksvolle Ergebnisse.
		Die Spannung am Kondensator konnten eindeutig als Integriertes Signal der Spannungsquelle identifiziert werden.
		Zudem lie"sen sich diese Ergebnisse sch"on visualisieren.

	\section{Literatur}

		Dieser Versuch wurde nach der Anleitung Das Relaxationsverhalten eines RC-Kreisesdurchgef"uhrt. Die Skizzen wurden hieraus nachgezeichnet.
		Alle Graphen und Screenshots wurden selbst erstellt. 

\end{document}
