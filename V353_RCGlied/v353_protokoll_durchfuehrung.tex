\documentclass{scrartcl}
	%ngerman fur Umlaute, Anf"hrungszeichen, etc.%
	\usepackage{ngerman}

	%siunitx fur SI-Einheitesnsystem%
	\usepackage{siunitx}
	\sisetup{
	    locale=DE,
	    separate-uncertainty=true,
    	per-mode=fraction
	}

	%graphicx fur Bildeinbindung%
	\usepackage{graphicx}

	%Texteinzug vor Absatz entfernen%
	\parindent0pt

\begin{document}

	\section{Aufbau und Durchf"uhrung}

		F"ur den Versuch stand ein RC-Glied, ein Oszillator, ein Millivoltmeter und ein digitales Zweikanaloszilloskop zu Verf"ugung.


		\subsection{Messaufgaben}

			\begin{enumerate}
				\item \label{aufg_1} Bestimmung der Zeitkonstante $\lambda$ des gegebenen RC-Gliedes

				durch Beobachtung des Auf- oder Entladevorganges des Kondensators
				\item \label{aufg_2} Messung der Amplitude der Kondensatorspannung $U_{\mathrm{C}}$ unter einer sinusf"ormigen Eingangsspannung $U_0$ in Abh"angigkeit der Frequenz $\omega$

				\item \label{aufg_3} Messung der Phasenverschiebung $\delta \phi$ zwischen Generatorspannung $U_0$ und Kon\-den\-sator\-spannung $U_{\mathrm{C}}$ in Abh"angigkeit der Frequenz $\omega$

				\item \label{aufg_4} Veranschaulichung des RC-Gliedes als Integrierglied, wenn $\omega >> \frac{1}{RC}$

			\end{enumerate}

		\subsection{Durchf"uhrung}

			F"ur die Bestimmung der Zeitkonstante $\lambda$ (Aufg. \ref{aufg_1}) wurde das RC-Glied an eine Rechteck\-spannung $U_0$ angeschlossen.
			Mit Hilfe des Oszilloskopes wurden dann Messwerte der Kondensatorspannung $U_{\mathrm{C}}$ zu verschiedenen Zeitpunkten $t$ aufgenommen.
			$t$ wurde dabei so gew"ahlt, dass alle Werte auf der Kurve eines Auf- bzw. Entladevorganges lagen,
			um daraus sp"ater die Zeitkonstante $\lambda$ zu ermitteln.
			Abbildung \ref{abb_aufbau_1} und \ref{abb_aufbau_2} zeigen den Versuchsaufbau und einen Screenshot des Oszilloskopes.\\

			\begin{figure}[ht]
				\centering
				\includegraphics[width = 2cm]{homer.png}
					\caption {Ohmscher Widerstand}
			\end{figure}

			Um die sperrende Eigenschaft des RC-Gliedes f"ur gro"se Frequenzen $\omega$ zu pr"ufen, haben wir nun die Kondensatorspannung $U_{\mathrm{C}}$ mit einem Millivoltmeter gemessen (Aufgabe \ref{aufg_2}).
			Der Funktions\-generator wurde auf eine Sinusspannung eingestellt und die Frequenz $\omega$ von $\SI{25}{\hertz}$ bis $\SI{5}{\kilo\hertz}$ variiert. Abbildung \ref{abb_aufbau_3} zeigt den neuen Versuchsaufbau. \\

			\begin{figure}[ht]
				\centering
				\includegraphics[width=1in, keepaspectratio=true]{pa.jpeg}
					\caption {Versuchsaufbau f"ur Amplitudenmessung an zwei Kondensatorplatten}
			\end{figure}

			Die Phasenverschiebung zwischen Generatorspannung und Kondensatorspannung konnte mit Hilfe des Zweikanaloszilloskopes gemessen werden (Aufgabe \ref{aufg_3}).
			Die Signale wurden auf dieselbe Nulllage gebracht und bei verschiedenen Frequenzen drei Messwerte $a_1$, $a_2$ und $b_1$ genommen.
			Aus diesen Werten konnten wir mit der Identit"at

			\begin{equation}
				\phi = \frac{a}{b} 2 \pi
			\end{equation}

			die Phasenverschiebung $\phi$ berechnen.
			Abbildung \ref{abb_ss_phase} zeigt die "ubereinandergelegten Signale $U_0$ und $U_{\mathrm{C}}$. Die Messpunkte wurden nachtr"aglich markiert.\\

			Schlie"slich haben wir eine Frequenz von $\SI{5}{\kilo\hertz}$ eingestellt, und somit $\omega >> \frac{1}{RC}$ gew"ahlt.
			Dadurch diente das RC-Glied als Integrator (siehe \ref{theorie_integrierglied}).
			Anhand von verschieden geformten Eingangssignalen $U_0$ konnte die Funktion sehr gut veranschaulicht werden (Aufgabe \ref{aufg_4}).
			Die Abbildungen \ref{abb_integriergl_1} bis \ref{abb_integriergl_4} zeigen die Ergebnisse.
			Es wurden Eingangsspannungen $U_0$ mit Sinus-, Dreieck- und Rechteckform eingestellt.

\end{document}
