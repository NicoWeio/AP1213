\newpage
\section{Diskussion}
	\label{sec:diskussion}

	Der Versuch hat gezeigt, dass Experiment und Theorie nah beieinander liegen.
	Die Abweichungen zwischen den errechneten und den gemessen Werten lagen zwischen $\SI{0.2}{\%}$ und $\SI{13}{\%}$.

	Besonders bei dem Ablesen des Widerstands f"ur den aperiodischen Grenzfall war es schwierig den exakten Wert zu ermitteln, da das Oszilloskop nicht sehr genau ablesbar ist.
	Dies gilt dementsprechend auch f"ur das Ablesen der Werte aus den Ausdrucken, da auch diese nicht genau sind.
	Der Versuch ist damit der anf"allig f"ur ablesefehler.

	F"ur die meisten Werte gab es jedoch gute "Ubereinstimmungen und der Versuch ist gut geeignet um ged"ampfte- und erzwungene Schwingungen am RLC Kreis zu untersuchen.

\section{Literaturverzeichnis}
	\label{sec:literaturverzeichnis}

	\begin{thebibliography}{9}
		\bibitem{anleitung} Physikalisches Anf"angerpraktikum der TU Dortmund: Versuch Nr. 354 - Ged"ampfte und erzwungene Schwingungen. Stand: Dezember 2012.
	\end{thebibliography}
