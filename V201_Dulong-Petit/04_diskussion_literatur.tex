\section{Diskussion}
\label{diskussion}

Die statistischen Fehler waren bei diesem Versuch relativ hoch. 

Das Thermoelement schwankte auch bei keiner Temperaturdifferenz einen Wert zwischen $\SI{-0.2}{\milli\volt}$ und $\SI{-0.15}{\milli\volt}$ an.

Dazu kommt die Annahme, dass keine Temperatur nach au"sen abgegeben wird. Dies war bereits zu sehen, als die Probe aus dem Wasserbad entnommen wurde, da sich bereits dort die Thermospannung "anderte.
Zudem ist das Gef"a"s nach oben nur durch einen Plastikdeckel verschlossen, wodurch zus"atzlich W"arme entweichen kann.

Die Spannung des Thermoelement schwankte beim ablesen der Spannung in den Nachkommastellen, wobei nur eine kleine "Anderung gen"ugen um das Endergebnis signifikant zu "andern.

So konnte der theoretische Wert f"ur $C_\mathrm{V} = 3R \approx 24.93$ nicht best"atigt werden. Die Werte f"ur Blei und Zinn lagen bei etwa $\SI{33.491 (633)}{\joule\per\kelvin\per\mol}$ und $\SI{30.609 (4225)}{\joule\per\kelvin\mol}$. Beide Werte liegen systematisch zu hoch.

Der Wert f"ur Graphit mit $\SI{6.002 (799)}{\joule\per\kelvin\per\mol}$ ist sogar nur 1/4 des theoretischen Werts. Dies steht nicht in Einklang mit dem Dulong-Petit'schen Gesetz und spricht f"ur einen Sonderfall bei hohen Temperaturen und gro"sen Atommassen. Dem entspricht, dass die Werte f"ur Zinn und Blein dichter an diesem Wert liegen, da diese eine h"ohere Atommasse haben als das aus Kohlenstoff bestehende Graphit.


\begin{thebibliography}{9}
	\bibitem{anleitung} Physikalisches Anf"angerpraktikum der TU Dortmund: Versuch V201 - Das Dulong-Petitsche Gesetz. Stand: Mai 2013.
\end{thebibliography}
