\section{Auswertung}
\label{sec:auswertung}

\subsection{Bestimmung der W"armekapazit"at des Kalorimeters} % (fold)
\label{sub:bestimmung_der_w_armekapazit_at_des_kalorimeters}

\begin{table}[!h]
\begin{center}
\begin{tabular}{|l|r|r|r|r|r|r|r|r|}
\hline
Probe & $U_\mathrm{kalt}$[$\SI{}{\milli\volt}$] & T[$^circ$] & $U_\mathrm{warm}$[$\SI{}{\milli\volt}$] & T[$^circ$] & $U_\mathrm{m}$[$\SI{}{\milli\volt}$] & T[$^circ$] & $m_\mathrm{warm}$[$\SI{}{\gram}$] & $m_\mathrm{kalt}$[$\SI{}{\gram}$]\\
\hline
\hline
Wasser & 0.89 & 22.24 & 2.55 & 62.91 & 1.68 & 41.73 & 301 & 300 \\
\hline
\end{tabular}
\caption[]{Messwerte zur Bestimmung der W"armekapazit"at des Kalorimeters.}
\label{kalo}
\end{center}
\end{table}

Bei der Bestimmung der W"armekapazit"at $c_\mathrm{g}m_\mathrm{g}$ des Kalorimeters ergaben sich die Gr"o"sen aus Tabelle \ref{kalo}.
Mithilfe der Gleichung \eqref{cg} ergibt sich f"ur $c_\mathrm{g}m_\mathrm{g}$:

\begin{equation}
	c_\mathrm{g}m_\mathrm{g} = \SI{113.88}{\joule\per\kelvin}
\end{equation}

Dabei werden die Temperaturen mittels der Gleichung \eqref{T} aus der Spannung errechnet.

\subsection{Bestimmung der spezifischen W"armekapazit"at bei Graphit, Blei und Zinn} % (fold)
\label{sub:bestimmung_der_spezifischen_w_armekapazit_at_bei_graphit_blei_und_zinn}


\begin{table}[!h]
\begin{center}
\begin{tabular}{|l|r|r|r|r|r|r|r|}
\hline
Probe & $U_\mathrm{w}$[$\SI{}{\milli\volt}$] & T[$^circ$] & $U_\mathrm{p}$[$\SI{}{\milli\volt}$] & T[$^circ$] & $U_\mathrm{m}$[$\SI{}{\milli\volt}$] & T[$^circ$] & $m_\mathrm{w}$[$\SI{}{\gram}$]\\
\hline
\hline
Graphit & 0.88 & 21.99 & 2.51 & 61.65 &	0.97 & 24.22 & 600 \\
Graphit & 0.91 & 22.74 & 3.45 &	84.53 & 1.00 & 24.97 & 601 \\
Graphit & 0.89 & 22.24 & 3.20 &	78.56 & 0.99 & 24.72 & 601 \\
\hline
Blei    & 0.93 & 23.23 & 2.67 & 65.81 &	1.01 & 25.21 & 601 \\
Blei    & 0.92 & 22.98 & 3.23 &	79.27 & 1.02 & 25.46 & 601 \\
Blei    & 0.93 & 23.23 & 3.20 &	78.56 & 1.03 & 25.71 & 600 \\
\hline
Zinn    & 0.91 & 22.74 & 3.50 &	85.72 & 1.03 & 25.71 & 600 \\
Zinn    & 0.94 & 23.48 & 3.01 &	74.00 & 1.01 & 25.21 & 601 \\
Zinn    & 0.95 & 23.73 & 3.18 & 78.08 & 1.02 & 25.46 & 600 \\
\hline
\end{tabular}
\caption[]{Messwerte zur Bestimmung der spezifischen W"armekapazit"at.}
\label{proben}
\end{center}
\end{table}

\begin{table}[!h]
\begin{center}
\begin{tabular}{|l|r|l|r|l|r|}
\hline
Probe & $C_\mathrm{k}$[$\SI{}{\joule\per\gram\kelvin}$] & Probe & $C_\mathrm{k}$[$\SI{}{\joule\per\gram\kelvin}$] & Probe & $C_\mathrm{k}$[$\SI{}{\joule\per\gram\kelvin}$]\\
\hline
\hline
Graphit & 0.625 & Blei & 0.176 & Zinn & 0.343 \\
Graphit & 0.396 & Blei & 0.166 & Zinn & 0.247 \\
Graphit & 0.487 & Blei & 0.168 & Zinn & 0.228 \\
\hline
\end{tabular}
\caption[]{Werte f"ur die spezifische W"armekapazit"at aus den Messungen von Graphit, Blei und Zinn.}
\label{kapa}
\end{center}
\end{table}

Die Daten f"ur die Messung der spezifischen W"armekapazit"at sind in Tabelle \ref{proben} aufgelistet.
Mit der Gleichung \eqref{ck} ergeben sich daraus und dem zuvor errechneten $c_\mathrm{g}m_\mathrm{g}$ die in Tabelle \ref{kapa} aufgelisteten Werte f"ur die spezifische W"armekapazit"at.
In Tabelle \ref{mittel} sind die Mittelwerte der spezifischen W"armekapazit"at abgebildet. Dabei ist zu ber"ucksichtigen, dass der Wert f"ur Graphit nicht besonders representativ ist, da die Werte recht weit auseinander liegen.


\begin{table}[!h]
\begin{center}
\begin{tabular}{|l|r|r|r|r|}
\hline
Probe & $\bar{c_\mathrm{k}}$[$\SI{}{\joule\per\gram\kelvin}$] & $\Delta\overline{c_\mathrm{k}}$[$\SI{}{\joule\per\gram\kelvin}$] & T[$\SI{}{\kelvin}$] & $\Delta$ T[$\SI{}{\kelvin}$] \\
\hline
\hline
Graphit & 0.503 & 0.067 & 297.77 & 0.22\\
Blei    & 0.170 & 0.003 & 298.59 & 0.14\\
Zinn    & 0.273 & 0.036 & 298.59 & 0.14\\
\hline
\end{tabular}
\caption[]{Mittelwerte der spezifischen W"armekapazit"at der verschiedenen Stoffe.}
\label{mittel}
\end{center}
\end{table}

\clearpage

\subsection{Bestimmung der Molw"arme $C_\mathrm{V}$ bei konstantem Volumen} % (fold)
\label{sub:bestimmung_der_molw_arme_c_mathrm}


\begin{table}[!h]
\begin{center}
\begin{tabular}{|l|r|r|r|r|}
\hline
Probe & $\rho$[$\SI{}{\gram\per\centi\meter^3}$] & M[$\SI{}{\gram\per\mol}$] & $\alpha$[$\SI{e-6}{1\per\kelvin}$] & $\kappa$ [$\SI{e9}{\newton\per\meter^2}$]\\
\hline
\hline
Graphit	& 2.25  & 12.0  & $\approx$ 8.0  & 33\\
Blei	& 11.35 & 207.2 & 29.0 & 42\\
Zinn	& 7.28  & 118.7 & 27.0 & 55\\
\hline
\end{tabular}
\caption[]{Dichte $\rho$, Molmasse M, linearer Ausdehnungskoeffizient $\alpha$ und Kompressionsmodul $\kappa$ der Proben.}
\label{massen}
\end{center}
\end{table}

F"ur die folgenden Berechnungen werden die Werte aus Tabelle \ref{massen} verwendet.
Die Molw"arme $C_\mathrm{P}$ bei konstantem druck l"asst sich aus der spezifischen W"armekapazit"at errechnen:

\begin{equation}
	C_\mathrm{P} = c_\mathrm{k} M \qquad .
\end{equation}

Aus dieser Gr"o"se ergibt sich f"ur die Molw"arme $C_\mathrm{V}$ bei konstantem Druck:

\begin{equation}
		C_\mathrm{V} = C_\mathrm{P} - 9 \alpha^2 \kappa \frac{M}{\rho} T
\end{equation}

Dabei wird f"ur $T$ die gemittelte Mischtemperatur eingesetzt.

Es ergeben sich f"ur die Molw"arme die in Tabelle \ref{warm} aufgelisteten Werte.


\begin{table}[!h]
\begin{center}
\begin{tabular}{|l|r|}
\hline
Probe & $C_\mathrm{P}$[$\SI{}{\joul\per\kelvin\mol}$] & $\Delta C_\mathrm{P}$[$\SI{}{\joul\per\kelvin\mol}$] & $C_\mathrm{V}$[$\SI{}{\joul\per\kelvin\mol}$] & $\Delta C_\mathrm{V}$[$\SI{}{\joul\per\kelvin\mol}$]\\
\hline
\hline
Graphit & 6.032  & 0.799 & 6.002  & 0.799\\
Blei    & 35.224 & 0.633 & 33.491 & 0.633\\
Zinn    & 32.366 & 4.225 & 30.609 & 4.225\\
\hline
\end{tabular}
\caption[]{Ergebnisse der Molw"armen $C_\mathrm{P}$ und $C_\mathrm{V}$ anhand der Messwerte.}
\label{warm}
\end{center}
\end{table}