\documentclass{scrartcl}
	%ngerman für Umlaute, Anführungszeichen, etc.%
	\usepackage{ngerman}

	%siunitx für SI-Einheitesnsystem%
	\usepackage{siunitx}

	%graphicx für Bildeinbindung%
	\usepackage{graphicx}

	%Texteinzug vor Absatz entfernen%
	\parindent0pt

\begin{document}

	\section{Einleitung}

	Dieser Versuch verdeutlicht die Funktionsweise eines Lock-In-Verst"arkers.
	Der Verst"arker zeichnet sich durch einen integrierten Phasenempfindlichen Detektor aus.
	Er ist somit in der Lage, stark verrauschte Signale mit gro"sen G"uten zu filtern.


	\section{Theorie}

	Der Lock-In-Verst"arker ist aus vier grundlegenden Bauteilen aufgebaut.
	Ein Bandpa"sfilter dient als Vorfilter. Ein Mischer multipliziert das gefilterte Signal mit einem Referenzsignal,
	das durch einen Phasenschieber mit dem Eingangssignal in Phase gebracht werden kann.
	Ein Tiefpa"sfilter dient schlie"slich als Integrierglied und gl"attet das Signal.

	Hiernach gilt f"ur das Ausgangssignal:

	\begin{equation}
	\label{gl1}
	U_{out} \propto U_o cos \phi (1)
	\end{equation}

	F"ur diesen Versuch wird als Eingangssignal ein Signal $U_0$ mit bekannter Frequenz $\omega_0$ benutzt und mit einem Rauschen versehen.
	Der Bandpa"sfilter filtert alle Frequenzen $\omega << \omega_0$ und $\omega >> \omega_0$ heraus.
	Danach wird $U_0$ mit einem Referenzsignal $U_{ref}$ mit konstanter Amplitude und der Frequenz $\omega_0 + \phi$ multipliziert ($\phi$ ist die Phasendifferenz zwischen $U_0$ und $U_{ref}$).
	Diese Frequenz kann durch den Phasenschieber mit $U_0$ synchronisiert werden ($\phi = 0$). Das so variierte Signal kann nun durch den Tiefpa"s integriert werden,
	wobei $\tau = RC >> 1/\omega_0$ gilt.
	Durch diese Integration wird das Signal gegl"attet, sodass die Identit"at \ref{gl1} erf"ullt ist.
	Damit l"asst sich die G"ute eines einfachen Bandpa"sfilters von q = 1 000 auf bis zu Q = 100 000 verbessern.


	Skizze Versuchsaufbau


	Folgendes Beispiel dient der Veranschaulichung. Es wird das Eingangssignal

	\begin{displaymath}
		U_{sig} = U_0 sin(\omega t)
	\end{displaymath}

	betrachtet. Wie oben beschrieben,
	wird dies nun mit einer Rechteckspannung als Refenzsignal multipliziert:

	\begin{eqnarray*}
		U_{ref} &=& \frac{4}{\pi}
			\left( \sin(\omega t + \phi) + 
				\frac{1}{3} \sin(3 \omega t + \phi) + 
				\frac{1}{5} \sin(5 \omega t + \phi) + 
				\cdots 
			\right)	\\
		U_{sig} \times U_{ref} &=& \frac{2}{\pi} U_0 
		\left(
			1 - 
			\frac{2}{3}\cos(2 \omega t + \phi) + 
			\frac{2}{15} \cos(4 \omega t + \phi) + 
			\frac{2}{35} \cos(6 \omega t + \phi) + 
			\cdots 
		\right)
	\end{eqnarray*}

	Nach der Intefration durch den Tiefpa"sfilter fallen alle $\cos$ - Anteile weg und wir erhalten eine zur Signalspannung proportionale Gleichspannung:


Man sieht, dass die Ausgangsspannung maximal wird, wenn Signal und Referenz in Phase sind.
 

	\begin{equation}
		U_{out} = \frac{2}{\pi} U_0 \cos{\phi}
	\end{equation}


	Man sieht, dass die Ausgangsspannung maximal wird, wenn Signal und Referenz in Phase sind.

	\section{Aufbau und Durchf"uhrung}

	Für diesen Versuch stand ein modular aufgebauter Lock-In-Verst"arker und ein Oszilloskop zur Verf"ugung.
	Der Verst"arker beinhaltete folgende unabhängige Module:

	\begin{itemize}
		\item Vorverst"arker
		\item Filter (Tiefpa"s, Hochpa"s, Bandpa"s)
		\item Amplituden-/Lock-In-Detektor
		\item Tiefpa"s Verst"arker (Integrierglied)
		\item Rauschgenerator
		\item Phasenschieber
		\item Funktionsgenerator (Oszillator)
	\end{itemize}
	Skizze (Aufbau Lock-In-Verst"arker)

\subsection{Messaufgaben}

\begin{enumerate}
\item \label{a0} Messen des Augsangssignals nach jedem Modul
\item \label{a1} Messen des Ausgangssignals nach dem Detektor ohne Rauschen mit verschiedenen Phasen
\item \label{a2} Messen des Ausgangssignals nach dem Integrierglied ohne Rauschen mit verschiedenen Phasen
\item \label{a3} Messen des Ausgangssignals nach dem Detektor mit Rauschen mit verschiedenen Phasen
\item \label{a4} Messen des Ausgangssignals nach dem Integrierglied mit Rauschen mit verschiedenen Phasen
\item \label{a5} Messen des Ausgangssignals mit verschiedenen Abständen der LED zum Photodetektor
\end{enumerate}

	Zun"achst wurden die Module schrittweise in die Schaltung eingebunden.
	Nach jeder neuen Konfiguration wurde das Ausgangssignal am Oszilloskop ausgegeben und gespeichert,
	um den Einfluss der einzelnen Module zu veranschaulichen . (Aufgabe \ref{a0})

	Hiernach wurden verschiedene Phasendifferenzen eingestellt und hinter dem Detektor gemessen.
	Dabei wurde der Rauschgenerator zun"achst "uberbr"uckt.(Aufgabe \ref{a1})
	Dies wurde hinter dem Integrierglied wiederholt.(Aufgabe \ref{a2})

	Im n"achsten Schritt wurde ein Rauschen in der Gr"o"senordnung der Signalspannung hinzugefügt und die Messungen
	wiederholt. (Aufgabe \ref{a3}, Aufgabe \ref{a4})

	Anschließend wurde der Rauschgenerator durch eine LED und eine Photozelle(PD) ersetzt.
	Die Signalfrequenz sollte auf $\SI{50}{\hertz}-\SI{500}{\hertz}$ eingestellt werden.
	Zus"atzlich wird die Signalspannung am Schalter von Sinus auf Rechtechsspannung umgestellt.
	Während der Messungen wurde der Abstand zwischen LED und PD vergrößert und so der Maximale Abstand ermittelt,
	bei dem das Licht noch nachgewiesen werden konnte. (Aufgabe \ref{a5})

\end{document}