\documentclass{scrartcl}
	%ngerman für Umlaute, Anführungszeichen, etc.%
	\usepackage{ngerman}

	%siunitx für SI-Einheitesnsystem%
	\usepackage{siunitx}
	\sisetup{
	    locale=DE,
	    separate-uncertainty=true,
    	per-mode=fraction
	}

	%graphicx für Bildeinbindung%
	\usepackage{graphicx}

	%Texteinzug vor Absatz entfernen%
	\parindent0pt

\begin{document}

	\section{Einleitung}

		Dieser Versuch verdeutlicht die Funktionsweise eines Lock-In-Verst"arkers.
		Der Verst"arker zeichnet sich durch einen integrierten Phasenempfindlichen Detektor aus.
		Er ist somit in der Lage, stark verrauschte Signale mit gro"sen G"uten zu filtern.


	\section{Theorie}

		Der Lock-In-Verst"arker ist aus vier grundlegenden Bauteilen aufgebaut.
		Ein Bandpa"sfilter dient als Vorfilter. Ein Mischer multipliziert das gefilterte Signal mit einem Referenzsignal,
		das durch einen Phasenschieber mit dem Eingangssignal in Phase gebracht werden kann.
		Ein Tiefpa"sfilter dient schlie"slich als Integrierglied und gl"attet das Signal.

		Hiernach gilt f"ur das Ausgangssignal:

		\begin{equation}
			\label{gl1}
			U_{out} \propto U_o cos \phi (1)
		\end{equation}

		F"ur diesen Versuch wird als Eingangssignal ein Signal $U_0$ mit bekannter Frequenz $\omega_0$ benutzt und mit einem Rauschen versehen.
		Der Bandpa"sfilter filtert alle Frequenzen $\omega << \omega_0$ und $\omega >> \omega_0$ heraus.
		Danach wird $U_0$ mit einem Referenzsignal $U_{ref}$ mit konstanter Amplitude und der Frequenz $\omega_0 + \phi$ multipliziert ($\phi$ ist die Phasendifferenz zwischen $U_0$ und $U_{ref}$).
		Diese Frequenz kann durch den Phasenschieber mit $U_0$ synchronisiert werden ($\phi = 0$). Das so variierte Signal kann nun durch den Tiefpa"s integriert werden,
		wobei $\tau = RC >> 1/\omega_0$ gilt.
		Durch diese Integration wird das Signal gegl"attet, sodass die Identit"at \ref{gl1} erf"ullt ist.
		Damit l"asst sich die G"ute eines einfachen Bandpa"sfilters von q = 1 000 auf bis zu Q = 100 000 verbessern.


		Skizze Versuchsaufbau

		\label{beispiel}
		Folgendes Beispiel dient der Veranschaulichung. Es wird das Eingangssignal

		\begin{displaymath}
			U_{sig} = U_0 sin(\omega t)
		\end{displaymath}

		betrachtet. Wie oben beschrieben,
		wird dies nun mit einer Rechteckspannung als Refenzsignal multipliziert:

		\begin{eqnarray*}
			U_{ref} &=& \frac{4}{\pi}
				\left( \sin(\omega t + \phi) + 
					\frac{1}{3} \sin(3 \omega t + \phi) + 
					\frac{1}{5} \sin(5 \omega t + \phi) + 
					\cdots 
				\right)	\\
			U_{sig} \times U_{ref} &=& \frac{2}{\pi} U_0 
			\left(
				1 - 
				\frac{2}{3}\cos(2 \omega t + \phi) + 
				\frac{2}{15} \cos(4 \omega t + \phi) + 
				\frac{2}{35} \cos(6 \omega t + \phi) + 
				\cdots 
			\right)
		\end{eqnarray*}

		Nach der Intefration durch den Tiefpa"sfilter fallen alle $\cos$ - Anteile weg und wir erhalten eine zur Signalspannung proportionale Gleichspannung:


		Man sieht, dass die Ausgangsspannung maximal wird, wenn Signal und Referenz in Phase sind.
	 

		\begin{equation}
			U_{out} = \frac{2}{\pi} U_0 \cos{\phi}
		\end{equation}


		Man sieht, dass die Ausgangsspannung maximal wird, wenn Signal und Referenz in Phase sind.

		\section{Aufbau und Durchf"uhrung}

		Für diesen Versuch stand ein modular aufgebauter Lock-In-Verst"arker und ein Oszilloskop zur Verf"ugung.
		Der Verst"arker beinhaltete folgende unabhängige Module:\label{module}

		\begin{itemize}
			\item Vorverst"arker
			\item Filter (Tiefpa"s, Hochpa"s, Bandpa"s)
			\item Amplituden-/Lock-In-Detektor
			\item Tiefpa"s Verst"arker (Integrierglied)
			\item Rauschgenerator
			\item Phasenschieber
			\item Funktionsgenerator (Oszillator)
		\end{itemize}
		Skizze (Aufbau Lock-In-Verst"arker)

		\subsection{Messaufgaben}

			\begin{enumerate}
				\item \label{a0} Messen des Augsangssignals nach jedem Modul
				\item \label{a1} Messen des Ausgangssignals nach dem Detektor ohne Rauschen mit verschiedenen Phasen
				\item \label{a2} Messen des Ausgangssignals nach dem Integrierglied ohne Rauschen mit verschiedenen Phasen
				\item \label{a3} Messen des Ausgangssignals nach dem Detektor mit Rauschen mit verschiedenen Phasen
				\item \label{a4} Messen des Ausgangssignals nach dem Integrierglied mit Rauschen mit verschiedenen Phasen
				\item \label{a5} Messen des Ausgangssignals mit verschiedenen Abständen der LED zum Photodetektor
			\end{enumerate}

		\subsection{Durchf"uhrung}

			Zun"achst wurden die Module schrittweise in die Schaltung eingebunden.
			Nach jeder neuen Konfiguration wurde das Ausgangssignal am Oszilloskop ausgegeben und gespei-chert,
			um den Einfluss der einzelnen Module zu veranschaulichen (Aufgabe \ref{a0}).

			Hiernach wurden verschiedene Phasendifferenzen eingestellt und hinter dem Detektor gemessen.
			Dabei wurde der Rauschgenerator zun"achst "uberbr"uckt (Aufgabe \ref{a1}).
			Dies wurde hinter dem Integrierglied wiederholt (Aufgabe \ref{a2}).

			Im n"achsten Schritt wurde ein Rauschen in der Gr"o"senordnung der Signalspannung hinzugefügt und die Messungen
			wiederholt (Aufgabe \ref{a3}, Aufgabe \ref{a4}).

			Anschließend wurde der Rauschgenerator durch eine LED und eine Photozelle(PD) ersetzt.
			Die Signalfrequenz sollte auf $\SI{50}{\hertz}-\SI{500}{\hertz}$ eingestellt werden.
			Zus"atzlich wird die Signalspannung am Schalter von Sinus auf Rechtechsspannung umgestellt.
			Während der Messungen wurde der Abstand zwischen LED und PD vergrößert und so der Maximale Abstand ermittelt,
			bei dem das Licht noch nachgewiesen werden konnte (Aufgabe \ref{a5}).

	\section{Auswertung}

		\subsection{Module des Verst"arkers}

		Die in \ref{module} aufgef"urten Module des Verst"arkers werden im Folgenden erl"autert.
		Durch das st"uckweise Hinzuschalten konnte der Einfluss auf das Eingangssignal beobachtet werden.

			\subsubsection{Funktionsgenerator (Oszilloskop)}

			Hier wird das Signal f"ur unseren Versuch generiert.
			Der Funktionsgenerator kann eine Sinus- oder Rechteckspannung im Bereich von $\SI{3}{\hertz}$ bis $\SI{3}{\kilo\hertz}$ ausgeben.
			Das Modul besitzt zudem zwei Ausg"ange. Der erste Ausgang liefert das zu messende Signal.
			Hier kann die Spannung von $\SI{1}{\volt}$ bis $\SI{10}{\volt}$ varriert und mit dem Faktor $\frac{1}{100}$ multipliziert werden.
			Der Zweite Ausgang liefert die Referenzsspannung, mit der das Signal im Detektor gemischt wird.
			Die Amplitude betrug bei unserem Ger"at konstante $\SI{23.2(5)}{\volt}$.
			Ausser-dem gibt der zweite Ausgang lediglich eine Sinusspannung aus.

			F"ur die Messungen \ref{a1} bis \ref{a3} haben wir eine Sinuspaunnung von $\SI{1}{\volt}$ und eine Frequenz von $\SI{1}{\kilo\hertz}$ eingestellt.
			Die Messung \ref{a4} wurde mit einer Rechteckspannung durchgef"uhrt.

			\subsubsection{Rauschgenerator}

			Das Signal aus dem Funktionsgenerator kann hier mit einem Rauschen belegt werden.
			Die Amplitude l"asst sich zwischen $\SI{e-5}{\volt}$ und $\SI{1}{\volt}$ einstellen.
			F"ur die Messungen \ref{a1} und \ref{a2} wurde der Rauschgenerator "uberbr"uckt. Die weiteren Messungen wurde mit einem Rauschen von $\SI{e-1}{\volt}$ durchgef"uhrt.

			\subsubsection{Vorverst"arker}

			Die Amplitude des Eingangssignales wird hier verst"arkt. Es k"onnen Verst"arkungen von 1 bis $e3$ eingestellt werden.
			In unseren Messreihen wurde dauerhaft die Verst"arkung 2 verwendet.

			\subsubsection{Filter}

			Hier wird das Signal aus dem Vorverst"arker gefiltert. Es stehen Tiefpa"s-, Hochpa"s- und Bandpa"sfilter zur Verf"ugung.
			Wir nutzten ausschlie"slich den Bandpa"sfilter. Durch Kombination eines Tiefpa"s- und Hochpa"sglied werden alle Frequenzen
			$\omega >> \omega_0$ und $\omega << \omega_0$ aus dem Signal eliminiert.
			$\omega_0$ kann hierbei im Frequenzspektrum des Funktions-generators (von $\SI{3}{\hertz}$ bis $\SI{3}{\kilo\hertz}$) gew"ahlt werden.
			Alle Frequenzen, die sich also stark von $\omega_0$ unterscheiden werden hier also gefiltert.
			Das Rauschen, welches wir in den Messungen \ref{a3} und \ref{a4} auf das Signal gelegt haben kann hierdurch stark reduziert werden.

			\subsubsection{Phasenschieber}

			Das Referenzsignal kann hier um einen Winkel $0$ bis $\pi$ verschoben werden.
			Dadurch kann das Eingangssignal $U_0$ mit dem Referenzsignal $U_{ref}$ in Phase gebracht werden,
			damit es im Mischer Gleichgerichtet wird.

			\subsubsection{Mischer (Lock-In-Detektor)}

			Der Mischer multipliziert das zu Messende Signal $U_0$ mit dem Referenzsignal $U_{ref}$ des Funktionsgenerators. Wenn beide Signale in Phase sind wird $U_0$ dadurch gleichgerichtet (siehe \ref{beispiel}).

			\subsubsection{Integrierglied (Tiefpa"s-Verst"arker)}

			Das letzte Modul in der Schaltung integriert das Signal aus dem Detektor "uber mehrere Perioden.
			Hierdurch erhalten wir ein Signal $U_{out} \propto U_0$, wie in \ref{theorie} erl"autert wurde.

			\subsection{LED-PD Schaltung}

			Der letzte Versuchsaufbau lieferte ein \"echt\" verrauschtes Signal.
			Die LED wurde in verschie-denen Abst"anden zu einem Photodetektor aufgebaut und mit einer Rechteckspannung gespeist.
			Durch das wei"se Umgebungslicht im Versuchsraum erreichte das Signal der LED den PD nur sehr verrauscht.
			Das Signal des PD wurde vom den Lock-In-Verst"arker detektiert.
			Wir erwarteten eine Abnahme des Signals im Verh"altnis $\frac{1}{r^2}$ mit dem Abstand $r$ zwischen LED und PD.
			Die LED wurde mit einer Frequenz von $\SI{125,25}{\hertz}$ betrieben. Die Messung ergab folgende Daten:


			Messdaten


			Tr"agt man die Daten auf einen Graphen, l"asst sich das erwartete verhalten leicht erkennen:

<<<<<<< HEAD

			Graph


	\section{Diskussion}
=======
\begin{enumerate}
\item \label{a0} Messen des Augsangssignals nach jedem Modul
\item \label{a1} Messen des Ausgangssignals nach dem Detektor ohne Rauschen mit verschiedenen Phasen
\item \label{a2} Messen des Ausgangssignals nach dem Integrierglied ohne Rauschen mit verschiedenen Phasen
\item \label{a3} Messen des Ausgangssignals nach dem Detektor mit Rauschen mit verschiedenen Phasen
\item \label{a4} Messen des Ausgangssignals nach dem Integrierglied mit Rauschen mit verschiedenen Phasen
\item \label{a5} Messen des Ausgangssignals mit verschiedenen Abständen der LED zum Photodetektor
\end{enumerate}

	Zun"achst wurden die Module schrittweise in die Schaltung eingebunden.
	Nach jeder neuen Konfiguration wurde das Ausgangssignal am Oszilloskop ausgegeben und gespeichert,
	um den Einfluss der einzelnen Module zu veranschaulichen . (Aufgabe \ref{a0})

	Hiernach wurden verschiedene Phasendifferenzen eingestellt und hinter dem Detektor gemessen.
	Dabei wurde der Rauschgenerator zun"achst "uberbr"uckt.(Aufgabe \ref{a1})
	Dies wurde hinter dem Integrierglied wiederholt.(Aufgabe \ref{a2})

	Im n"achsten Schritt wurde ein Rauschen in der Gr"o"senordnung der Signalspannung hinzugefügt und die Messungen
	wiederholt. (Aufgabe \ref{a3}, Aufgabe \ref{a4})

	Anschließend wurde der Rauschgenerator durch eine LED und eine Photozelle(PD) ersetzt.
	Die Signalfrequenz sollte auf $\SI{50}{\hertz}-\SI{500}{\hertz}$ eingestellt werden.
	Zus"atzlich wird die Signalspannung am Schalter von Sinus auf Rechtechsspannung umgestellt.
	Während der Messungen wurde der Abstand zwischen LED und PD vergrößert und so der Maximale Abstand ermittelt,
	bei dem das Licht noch nachgewiesen werden konnte. (Aufgabe \ref{a5})
>>>>>>> d7517faf8dd9c903bcc381e29fafa614d2042290

\end{document}