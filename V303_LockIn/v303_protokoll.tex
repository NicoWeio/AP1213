\documentclass{scrartcl}
\usepackage{ngerman}
\usepackage{siunitx}

\parindent0pt

\begin{document}

\section{Einleitung}

Dieser Versuch verdeutlicht die Funktionsweise eines Lock-In-Verst"arkers.
Der Verst"arker zeichnet sich durch einen integrierten Phasenempfindlichen Detektor aus.
Er ist somit in der Lage, stark verrauschte Signale mit gro"sen G"uten zu filtern.


\section{Theorie}

Der Lock-In-Verst"arker ist aus vier grundlegenden Bauteilen aufgebaut.
Ein Bandpa"sfilter dient als Vorfilter. Ein Mischer multipliziert das gefilterte Signal mit einem Referenzsignal,
das durch einen Phasenschieber mit dem Eingangssignal in Phase gebracht werden kann.
Ein Tiefpa"sfilter dient schlie"slich als Integrierglied und gl"attet das Signal.

Hiernach gilt f"ur das Ausgangssignal:

\begin{equation}
\label{gl1}
U_{out} \propto U_o cos \phi (1)
\end{equation}

F"ur diesen Versuch wird als Eingangssignal ein Signal $U_0$ mit bekannter Frequenz $\omega_0$ benutzt und mit einem Rauschen versehen.
Der Bandpa"sfilter filtert alle Frequenzen $\omega << \omega_0$ und $\omega >> \omega_0$ heraus.
Danach wird $U_0$ mit einem Referenzsignal $U_{ref}$ mit konstanter Amplitude und der Frequenz $\omega_0$ multipliziert.
Diese Frequenz kann durch den Phasenschieber mit $U_0$ synchronisiert werden. Das so variierte Signal kann nun durch den Tiefpa"s integriert werden,
wobei $\tau = RC >> 1/\omega_0$ gilt.
Durch diese Integration wird das Signal gegl"attet, sodass die Identit"at \ref{gl1} erf"ullt ist.
Damit l"asst sich die G"ute eines einfachen Bandpa"sfilters von q = 1 000 auf bis zu Q = 100 000 verbessern.

Skizze Versuchsaufbau

Im folgenden Beispiel wird das Eingangssignal

\begin{displaymath}
U_{sig} = U_0 sin(\omega t) \nonumber
\end{displaymath}

betrachtet.

\end{document}