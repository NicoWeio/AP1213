\clearpage

\section{Diskussion}
\label{sec:diskussion}
	Zun"achst muss festgehalten werden, dass die Durchf"uhrung dieses Versuchs einige Schwierigkeiten beinhaltet.
	Bei Drehung des Fernrohrs, wurde das Prisma oft unbeabsichtigt mitgedreht, was teilweise gro"se Messfehler zur Folge hatte und weshalb eine Messung wiederholt werden musste.
	
	Der Wert des Prismainnenwinkels $\varphi = \SI{71}{\degree}$ stimmt offensichtlich nicht mit dem tats"achlichen Aufbau des Prismas "uberein und weist damit auf einen Systematischen Fehler des Aufbaus hin.

	Die daraus basierenden Werte f"ur den Brechungsindex weichen dennoch nur etwa um $\SI{10}{\percent}$ von den mit Hilfe des tats"achlichen Prismenwinkels berechneten Werten ab.
	Die Werte des Brechungsindex selbst stimmen relativ gut mit dem erwarteten Wert von etwa $n = 2$ "uberein.

	Dasselbe gilt f"ur die Abbesche Zahl $\nu$.
	Mit einem Wert von $\nu = \SI{52.6(175)}{}$ liegt sie unter Ber"ucksichtigung des Fehlers innerhalb des Erwarteten Wertebereichs zwischen $\nu = 30$ und $\nu = 60$ \cite{abbe}.

	Das Aufl"osungsverm"ogen $A$ des Prismas liegt ebenfalls in der erwarteten Gr"o"senordnung von $A = \SI{2000}{}$, kann aber mangels Herstellerangaben nicht direkt verglichen werden.

	Die n"achste Absorptionsstelle $\lambda_1 = \SI{111}{\nano \meter}$ liegt erwartungsgem"a"s im Bereich des Ultravioletten.


\begin{thebibliography}{9}
	\bibitem{anleitung} Physikalisches Anf"angerpraktikum der TU Dortmund: Versuch V402 - Dispersion am Glasprisma. \url{http://129.217.224.2/HOMEPAGE/PHYSIKER/BACHELOR/AP/SKRIPT/V402.pdf}. Stand: Mai 2013.

	\bibitem{abbe} Carl Zeiss Vision GmbH. Grundlagen -- Optik \& Auge -- Abbe-Zahl. \url{http://www.zeiss.de/4125680F0052EC92/Contents-Frame/40FF5667B79376E841256865003C3E85}. Stand: Mai 2013
\end{thebibliography}
