\section{Diskussion}
\label{diskussion}

	Bei diesem Versuch gab es wenig Probleme. Das Ablesen der Fourieramplituden am Oszilloskop war bei der S"aegezahnspannung in h"oheren Oberwellenbereichen etwas schwietig, da das Signal sehr verrauscht war.
	Bei dem Erzeugung von Schwingungsformen mithilfe des Oberwellengenerators musste sich erst etwas eingearbeitet werden, doch nachdem man die richtigen Amplituden den richtigen Oberwellen zugeordnet hatte, hat auch dies gut funktioniert.
	Da die Phase bei $\SI{0}{^\circ}$ und $\SI{180}{^\circ}$ gleich aussah, lie"s sich diese erst durch ausprobieren richtig ermitteln.

\begin{thebibliography}{9}
	\bibitem{anleitung} Physikalisches Anf"angerpraktikum der TU Dortmund: Versuch V351 - Fourier-Analyse und Synthese. \url{http://129.217.224.2/HOMEPAGE/PHYSIKER/BACHELOR/AP/SKRIPT/V351.pdf}. Stand: Mai 2013.
\end{thebibliography}
