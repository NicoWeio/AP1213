
\begin{table}[!h]
\begin{center}
\begin{tabular}{|r|r|r|r|r|r|}
\hline
Rechteck & & Dreieck & & S"agezahn & \\
n-te Oberschwingugn & U[V] & n-te Oberschwingugn & U[V] & n-te Oberschwingugn & U[V]\\
\hline
\hline
1 & 0.600 & 1 & 0.600 & 1 & 0.600\\
2 & 0.000 & 2 & 0.000 & 2 & -0.300\\
3 & 0.200 & 3 & 0.067 & 3 & 0.200\\
4 & 0.000 & 4 & 0.000 & 4 & -0.150\\
5 & 0.120 & 5 & 0.024 & 5 & 0.120\\
6 & 0.000 & 6 & 0.000 & 6 & -0.100\\
7 & 0.085 & 7 & 0.012 & 7 & 0.085\\
8 & 0.000 & 8 & 0.000 & 8 & -0.075\\
9 & 0.065 & 9 & 0.007 & 9 & 0.065\\
\hline
\end{tabular}
\caption[]{Werte der theoretisch errechneten Amplituden aus den Fourier-Koeffizienten.}
\label{tab:amplituden}
\end{center}
\end{table}