
\begin{table}[!h]
\begin{center}
\begin{tabular}{|r|r|r|r|r|r|}
\hline
 Rechteck & & Saegezahn & & Dreieck & \\
 n-te Oberwelle & U[V] & n-te Oberwelle & U[V] & n-te Oberwelle & U[V] \\
\hline
\hline
1	& 3.700 & 1	& 1.880 & 1	& 2.220 \\
3	& 1.160 & 2	& 0.960 & 3	& 0.266 \\
5	& 0.740 & 3	& 0.640 & 5	& 0.092 \\
7	& 0.500 & 4	& 0.480 & 7	& 0.041 \\
9	& 0.400 & 5	& 0.392 & 9	& 0.024 \\
11	& 0.328 & 6	& 0.328 & 11& 0.018 \\
13	& 0.248 & 7	& 0.296 & 13& 0.014 \\
15	& 0.249 & 8	& 0.256 & 15& 0.011 \\
17	& 0.208 & 9	& 0.240 & 17& 0.010 \\
\hline
\end{tabular}
\caption[]{Messwerte der Spannung $U$ bei der n-ten Oberschwingung}
\label{tab:messwerte}
\end{center}
\end{table}