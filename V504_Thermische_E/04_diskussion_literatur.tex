\section{Diskussion}
	\label{diskussion}
	Besonders auff"allig sind in diesem Experiment die Abweichung der Kathodentemperatur zwischen Kapitel \ref{subsubsec:temp1} und \ref{subsubsec:temp2}.
	Der Unterschied von "uber $\SI{30}{\percent}$ l"asst sich nur duch Systematische Fehler erkl"aren.

	Zudem konnte die Austrittsarbeit auf Grund der kleinen Wertestichprobe nicht sonderlich genau bestimmt werden.
	Falls mehr Messwerte oberhalb von Kathodenspannungen von $U = \SI{90}{\volt}$ aufgenommen werden k"onnten, w"are die Abweichung von fast $\SI{70}{\percent}$ vom Literaturwert $e_0 \Phi = \SI{4.1}{\electronvolt}$ eventuell geringer ausgefallen.

\section{Literaturverzeichnis}
	\label{sec:literaturverzeichnis}

	\begin{thebibliography}{9}
		\bibitem{anleitung} Physikalisches Anf"angerpraktikum der TU Dortmund: Versuch Nr. 504 - Thermische Elektronenemission. Stand: Januar 2013.

		\bibitem{nist} National Institute of Standards and Technology: Reference on Constants, Units and Uncertainty. http://physics.nist.gov/cuu/index.html. Stand: 16.01.2013.
	\end{thebibliography}
