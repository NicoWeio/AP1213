\newpage
\section{Diskussion}
	\label{sec:diskussion}
	Dieser Versuch sollte eine Anwendungsm"oglichkeit von R"ontgenspektren beleuchten.

	Bei der Auswertung der Daten stellt sich heraus, dass die Messwerte stark von Literaturwerten abweichen. Besonders die Die Messung der L-Kanten liefert gro"se Unterschiede, die nur durch eine fehlerhafte Messung zu erkl"aren sind.

	\begin{table}
		\centering
		\caption{Werte aus Messung im Vergleich mit Literaturwerten \cite{literatur}}
		\begin{tabular}{|c|c|c|c|}
			\hline
			Element & Messwert & Literaturwert & Abweichung [\%] \\
			\hline
			\hline
			${}_{32}^{}\mathrm{Ge}$ & $\sigma_{1,0} = \SI{4.11 (8)}{}$ & $\sigma_{1,0} = \SI{3.43}{}$ & 19,8 \\
			${}_{41}^{}\mathrm{Nb}$ & $\sigma_{1,0} = \SI{4.16 (18)}{}$ & $\sigma_{1,0} = \SI{3.64}{}$ & 14,3 \\
			${}_{79}^{}\mathrm{Au}$ & $s_{2,1} = \SI{15.64 (8)}{}$ & $s_{2,1} = \SI{3.75}{}$ & 317,0 \\
			${}_{80}^{}\mathrm{Hg}$ & $s_{2,1} = \SI{17.34 (11)}{}$ & $s_{2,1} = \SI{3.77}{}$ & 359,9 \\
			\hline
		\end{tabular}		
	\end{table}

	Das erste gro"se Problem bestand in der Identifizierung der Absorptionskanten.
	Wie in Abbildung \ref{fig:germanium} unschwer zu erkennen ist, konnte man einen pl"otzlichen Abfall der Intensit"at nicht ohne weiteres erkennen.
	Ohne die vorherige Kenntnis "uber die Lage der Kanten lassen sich lediglich willk"urliche Ergebnisse erzielen.

	Zudem konnten systematische Fehler nicht ausgeschlossen werden.
	
\begin{thebibliography}{9}
	\bibitem{anleitung} Physikalisches Anf"angerpraktikum der TU Dortmund: Versuch Nr. 602 - R"ontgen-Emissions- und Absorptionsspektren. Stand: Oktober 2012.
	\bibitem{literatur} Biomolecular Structure Center der Universit"at von Washington, Seattle. http://www.bmsc.washington.edu/scatter/periodic-table.html
\end{thebibliography}