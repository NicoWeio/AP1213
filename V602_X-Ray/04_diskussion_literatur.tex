\newpage
\section{Diskussion}
	\label{sec:diskussion}
	Dieser Versuch sollte verschiedene Aspekte von R"ontgenspektren beleuchten. Der theoretische Hintergrund war zum einen sehr interessant, setzte jedoch quantenmechanische Vorkenntnisse voraus, die im Anf"angerpraktikum nicht unbedingt vorhanden sind.
	Hierduruch wurde der Zugang zum Versuch etwas erschwert.

	Die Durchf"uhrung hingegen war ausgesprochen Anwenderfreundlich, da keine komplizierten Einstellungen vorgenommen werden mussten.

	Bei der Auswertung der Daten stellte sich dann heraus, dass einige Messwerte stark von Literaturwerten abwichen.
	Das erste gro"se Problem bestand in der identifizierung der Absorptionskanten.
	Wie in Abbildung \ref{fig:germanium} unschwer zu erkennen ist, konnte man einen pl"otzlichen Abfall der Intensit"at nicht ohne weiteres erkennen.
	Ohne die vorherige Kenntnis "uber die Lage der Katen waren recht willk"urliche Ergebnisse erzielt worden.

	Zudem konnten systematische Fehler nicht ausgeschlossen werden.
	
\begin{thebibliography}{9}
	\bibitem{anleitung} Physikalisches Anf"angerpraktikum der TU Dortmund: Versuch Nr. 602 - R"ontgen-Emissions- und Absorptionsspektren. Stand: Oktober 2012.
\end{thebibliography}