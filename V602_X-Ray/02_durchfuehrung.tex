\section{Aufbau und Durchf"uhrung}
	\label{sec:durchfuehrung}
	F"ur diesen Versuch durchl"auft das R"ontgenlicht eine Blende, wodurch ein Strahl erzeugt wird.
	Dieser trifft auf einen Drehbaren Kristall mit Gitternetzabstand $d = \SI{201}{\pico \meter}$.
	Hier wird der Strahl unter Braggbedingungen reflektiert, wobei lediglich die Interferenzordnung $n = 1$ auftritt.
	Der reflektierte und gebeugte Strahl wird schlie"slich von einem Impulsratemeter detektiert.
	An diesem lassen sich zus"atzlich die zu untersuchenden Materialien befestigen.

	Das so gewonnene Signal kann nun am PC als Intensit"at $I$ gegen den Winkel $\theta$ aufgetragen werden.
	Es ist zu beachten, dass dieses Signal lediglich eine Mittelung "uber etliche Quanten ist.
	Pl"otzliche Impuls"anderungen, die z.B. an den Absorptionskanten auftreten werden somit nicht scharf aufgel"ost.

	\subsection{Messaufgaben}
		\begin{itemize}
			\item{Des energetischen Aufl"osungsverm"ogen der Apparatur.}
			\item{Abschirmzahl $\sigma_{1,0}$ von ${}_{32} \mathrm{Ge}$ und ${}_{41} \mathrm{Nb}$ aus den K Absorptionskanten.}
			\item{Abschirmzahl $s_{2,1}$ von ${}_{79} \mathrm{Au}$} und ${}_{80} \mathrm{Hg}$ aus den $\mathrm{L}_\mathrm{II}$ und $\mathrm{L}_\mathrm{II}$ Absorptionskanten.}
			\item{Sch"atzung der Abschirmzahlen $\sigma_{1}$ und $\sigma_{2}$ f"ur ${}_{28} \mathrm{Cu}$ ohne Ber"ucksichtigung des Drehimpulsbeitrages aus den Emissionsenergien $\mathrm{L}_\mathrm{K_\mathrm{\alpha}}$ und $\mathrm{L}_\mathrm{K_\mathrm{\beta}}$.}
		\end{itemize}