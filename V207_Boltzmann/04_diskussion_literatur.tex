\newpage
\section{Diskussion}

	Der Versuch hat die gew"unschten Ergebnisse geliefert.
	Wie erwartet wurde von der schwarzen und der wei"sen Oberfl"ache am meisten absorbiert.
	Diese bestehen aus Ma\-te\-ri\-a\-li\-en, welche besonders viel Strahlung absorbieren.
	Die matte und die spiegelnde Oberfl"ache hingegen kann mehr Strahlung reflektieren, wodurch das Emissionsverm"ogen sehr viel h"oher ist.

	Da wir das Emissionsverm"ogen der schwarzen Oberfl"ache mit $\epsilon = 1$ gesetzt wurde, obwohl es diesen Wert nicht in der Natur gibt,
	sind die Emissionsverm"ogen der anderen Oberfl"achen nicht ganz exakt. Daher ist es auch schwierig die Werte mit Literaturwerten zu vergleichen, doch liegen diese im Allgemeinen nach \cite{emission} unter den gemessenen Werten.

	Bei der Messung der Thermospannung bei verschiedenen Abst"anden ergab den ver\-mu\-te\-ten Leistungsabnahmezusammenhang mit $1/r^2$.

\begin{thebibliography}{9}
	\bibitem{anleitung} Physikalisches Anf"angerpraktikum der TU Dortmund: Versuch Nr. 207 - Das Stefan- Boltzmann Gesetz. Stand: November 2012.
	\bibitem{emission} http://www.omega.com/literature/transactions/volume1/emissivityb.html
\end{thebibliography}