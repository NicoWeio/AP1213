\section{Diskussion}
	\label{diskussion}

	Bei diesem Versuch ist die Br"uckenschaltung eine systematische Fehlerquelle.
	Durch Raufdr"ucken auf das Schaltelement oder auch durch Drehen der Probe, ver"anderte sich der erhaltene Wert f"ur die Spannung signifikant.
	Dies k"onnte an einer "Anderung der Leitf"ahigkeit der Spule liegen oder auch durch einen Wackelkontakt innerhalb des Elements.

	Die Messung der Filterkurve hat gut funktioniert und die gemessene G"ute weicht nur um $\SI{8.8}{\%}$ vom Theoriewert ab.

	Die errechneten Werte der Suszeptibilit"at f"ur Dysprosium und Gadolinium sind relativ niedrig, doch ist die Abweichung bei Neodym signifikant gr"o"ser.
	Da die Spannung $U_\mathrm{br}$ und der Querschnitt $A_\mathrm{real}$ des Neodyms sehr viel kleiner als die der anderen Materialien ist, fallen systematische Fehler noch st"arker ins Gewicht.

	Dieser Versuch ist daher sehr empfindlich, weil die Suszeptibilit"aten klein sind und systematische Fehler das Ergebnis relativ stark beeinflussen.


\begin{thebibliography}{9}
	\bibitem{anleitung} Physikalisches Anf"angerpraktikum der TU Dortmund: Versuch Nr. 606 - Messung der Suszeptibilit"at paramagnetischer Substanzen. Stand: Januar 2013.

	% \bibitem{nist} National Institute of Standards and Technology: Reference on Constants, Units and Uncertainty. http://physics.nist.gov/cuu/index.html. Stand: 16.01.2013.
\end{thebibliography}
