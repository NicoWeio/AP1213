\section{Diskussion}
	\label{diskussion}

	Bei diesem Versuch ist die Br"uckenschaltung eine systematische Fehlerquelle.
	Durch Raufdr"ucken auf das Schaltelement oder auch durch drehen der Probe, ver"anderte sich der erhaltene Wert f"ur die Spannung signifikant.
	Dies k"onnte an einer "anderung der Leitf"ahigkeit der Spule liegen oder auch durch einen Wackelkontakt innerhalb des Elements.

	Die Messung der Filterkurve hat gut funktioniert und die gemessene G"ute wich nur um $\SI{8.8}{\%}$ vom Theoriewert ab.

	Die errechneten Werte f"ur die Suszebilit"at aus dem Experiment liegen relativ dicht bei, w"ahrend sie bis zu 100 \% vom Theoriwert abweichen.
	Dies l"asst auf systematische Fehler innerhalb der Messapparatur schlie"sen.

	Dieser Versuch ist daher sehr empfindlich, gerade weil die Suszeptibilit"aten klein sind.


\begin{thebibliography}{9}
	\bibitem{anleitung} Physikalisches Anf"angerpraktikum der TU Dortmund: Versuch Nr. 606 - Messung der Suszeptibilit"at paramagnetischer Substanzen. Stand: Januar 2013.

	% \bibitem{nist} National Institute of Standards and Technology: Reference on Constants, Units and Uncertainty. http://physics.nist.gov/cuu/index.html. Stand: 16.01.2013.
\end{thebibliography}
