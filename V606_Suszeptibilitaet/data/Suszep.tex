
\begin{table}[!h]
\begin{center}
\begin{tabular}{|l|r|r|r|r|r|}
\hline
 &&&&&\\
 Probe & $\chi_\mathrm{exp1}$ Widerstand & $\chi_\mathrm{exp2}$ Spannung & $\chi_\mathrm{theo}$ & $\Delta \chi_\mathrm{1}$ & $\Delta \chi_\mathrm{2}$ \\
\hline
\hline

Dysprosium & $\SI{0.0239 (14)}{}$ & $\SI{0.0234 (1)}{}$ & $\SI{0.0253}{}$ & $\SI{5.86}{\%}$ & $\SI{8.12}{\%}$\\
Neodym     & $\SI{0.0026 (21)}{}$ & $\SI{0.0010 (1)}{}$ & $\SI{0.0030}{}$ & $\SI{15.38}{\%}$ & $\SI{200}{\%}$ \\
Gadolinium & $\SI{0.0110 (14)}{}$ & $\SI{0.0101 (1)}{}$ & $\SI{0.0137}{}$ & $\SI{24.55}{\%}$ & $\SI{35.64}{\%}$\\

\hline
\end{tabular}
\caption{Werte f"ur die Suszeptibilit"at mit 3 verschiedenen Messtechniken. $\chi_\mathrm{exp1}$ nach Gl. \eqref{chi_r}, $\chi_\mathrm{exp2}$ nach Gl. \eqref{chi_u} und $\chi_\mathrm{theo}$ nach Gl. \eqref{curie}}
\label{suszep}
\end{center}
\end{table}