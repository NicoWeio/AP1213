
\begin{table}[!h]
\begin{center}
\begin{tabular}{|l|r|r|r|r|r|}
\hline
 &&&&&\\
 Probe & $\chi_\mathrm{exp1}$ & $\chi_\mathrm{exp2}$ & $\chi_\mathrm{theo}$ & $\Delta \chi_\mathrm{1}$ & $\Delta \chi_\mathrm{2}$ \\
\hline
\hline

Dysprosium & $\SI{0.0239 (14)}{}$ & $\SI{0.0250 (1)}{}$ & $\SI{0.0126}{}$ & $\SI{89.68}{\%}$ & $\SI{98.41}{\%}$\\
Neodym     & $\SI{0.0026 (21)}{}$ & $\SI{0.0030 (1)}{}$ & $\SI{0.0015}{}$ & $\SI{73.33}{\%}$ & $\SI{100.00}{\%}$ \\
Gadolinium & $\SI{0.0110 (14)}{}$ & $\SI{0.0115 (1)}{}$ & $\SI{0.0069}{}$ & $\SI{59.42}{\%}$ & $\SI{66.66}{\%}$\\

\hline
\end{tabular}
\caption{Werte f"ur die Suszeptibilit"at mit 3 verschiedenen Messtechniken. $\chi_\mathrm{exp1}$ nach Gl. \eqref{chi_r}, $\chi_\mathrm{exp2}$ nach Gl. \eqref{chi_u} und $\chi_\mathrm{theo}$ nach Gl. \eqref{curie}}
\label{suszep}
\end{center}
\end{table}