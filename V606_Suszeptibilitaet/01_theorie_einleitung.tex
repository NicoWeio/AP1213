\section{Einleitung} % (fold)
	\label{sec:einleitung}

	Alle Stoffe haben die Eigenschaft, Magnetfelder, die sie umgeben zu beeinflussen.
	Zun"achst wird das Feld geschw"acht, was als Diamagnetismus bezeichnet wird.
	Bei einigen Stoffen tritt ein verst"arkender Effekt ein, der den Diamagnetismus zum Teil weit "ubertrifft.
	Dies ist der hier behandelte Paramagnetismus.

	In Welcher Art das Magnetfeld ver"andert wird, wird dabei durch die Suszeptibilit"atskonstante $\chi$ beschrieben. Diese wird im Folgenden Versuch f"ur verschiedene stark paramagnetische seltene Erden untersucht.

\section{Theorie} % (fold)
	\label{sec:theorie}

	Die Magnetische Flussdichte $\vec{B}$ ist mit der magentischen Feldst"arke $\vec{H}$ "uber

	\begin{equation*}
		\vec{B} = \mu_0 \left(1 + \chi\right) \vec{H}
	\end{equation*}

	verkn"upft.
	Dabei ist $\chi$ keinesfalls konstant, sondern h"angt von der Temperatur $T$ und der Beschaffenheit des Feldes $\vec{H}$ ab.

	\subsection{Berechnung der Suszeptibilit"at}
		\label{subsec:berechnung}
		Ein Atom, Ion oder Molek"ul mit nicht verschwindendem Drehimpuls ist in der Lage, sich an einem "au"seren Magnetfeld auszurichten.
		Der Gesamtdrehimpuls $\vec{J}$ setzt sich dabei aus den Anteilen des Bahndrehimpulses $\vec{L}$, des Gesamtspins $\vec{S}$ und des zu vernachl"assigenden Kerndrehimpulses ab:

		\begin{equation*}
			\vec{J} = \vec{L} + \vec{S}\,.
		\end{equation*}

		Zu den Drehimpulsen $\vec{L}$ und $\vec{S}$ geh"oren, entsprechend der Quantenmechanik, die magnetischen Momente $\vec{\mu}_\mathrm{L}$ und $\vec{\mu}_\mathrm{S}$.
		Nach einiger Rechnung f"uhren diese Gr"o"sen auf den gen"aherten Betrag des gesamten magnetischen Moments:

		\begin{equation*}
			\vec{\mu}_\mathrm{J} = \mu_\mathrm{B} g_\mathrm{J} \sqrt{J(J + 1)}\,.
		\end{equation*}

		Dabei bezeichnet $g_\mathrm{J}$ den Land\'e-Faktor, der die Gesamtdrehimpulsquantenzahl $J$, die Spinquantenzahl $S$ und die Bandrehimpulsquantenzahl $L$ des Atoms beinhaltet:

		\begin{equation*}
			g_\mathrm{J} = \frac{3 J(J + 1) + \{S (S + 1) - L(L + 1)\}}{2J(J + 1)}\,.
		\end{equation*}

		Der Zeeman-Effekt beschreibt nun, dass die Richtung, in die $\vec{\mu}_\mathrm{J}$ zeigt nur bestimmte Winkel zum Magnetfeld einnimmt, also ebenfalls gequantelt ist.
		Jede Richtung entspricht dabei einem Energieniveau und es l"asst sich durch Summation "uber alle Niveaus das mittlere Magnetische Moment bestimmen.
		Die gesuchte Gr"o"se $\chi$ kann dann f"ur die Annahme gro"ser Temperaturen - etwa Zimmertemperatur -  gen"ahert werden.
		\clearpage
		Man erh"alt schlie"slich das Curiesche Gesetz des Paramagnetismus:

		\begin{equation}
			\label{curie}
			\chi = \frac{\mu_0 \mu_\mathrm{B}^2 g_\mathrm{J}^2 N J(J+1)}{3kT}\,
		\end{equation}

		mit dem Bohrschen Magneton $\mu_\mathrm{B} = e_0 \hbar / 2 m_0$, das die Elektronenladung $e_0$ und -masse $m_0$, sowie das Plancksche Wirkungsquantum $\hbar$ beinhaltet, der Anzahl $N$ der Momente pro Volumen und der Boltzmannkonstante $k$.

		Weil dabei offensichtlich

		\begin{equation*}
			\chi \propto \frac{1}{T}
		\end{equation*}

		gilt, ist bei der Untersuchung der Suszeptibilit"at darauf zu achten, dass Temperaturschwankungen in den Stoffen vermieden werden.


