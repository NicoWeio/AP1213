\section{Aufbau und Durchf"uhrung}
	\label{sec:durchfuehrung}

	Die Messapparatur befindet sich auf einer optischen Bank, auf denen die optischen Elemente verschoben werden k"onnen.
	Die Lichtquelle ist eine Halogenlampe und der Gegenstand ein "`Perl L"'.
	Vorweg wird die Gr"o"se von "`Perl L"' bestimmt.

	\subsection{Bestimmung der Brennweite durch Messung der Gegenstandsweite und Bildweite} % (fold)
	\label{sub:bestimmung_der_brennweite_durch_messung_der_gegenstandsweite_und_bildweite}
	
	Zuerst wird die Brennweite von einer Linse mit bekannter Brennweite und einer mit unbekannter durchgef"uhrt.
	Daf"ur wird auf die optische Bank eine Halogenlampe, der Gegenstand "`Perl L"' , eine Sammellinse unbekannter/bekannter Brennweite und ein Schirm angebracht.

	Nach Ausrichten der Linse auf eine feste Gegenstandsweite $g$ wird der Schirm so eingestellt, dass das Bild scharf abgebildet wird.

	Das Wertepaar ($g_\mathrm{i}$, $b_\mathrm{i}$) wird f"ur insgesammt 10 verschiedene Gegenstandsweiten notiert.

	