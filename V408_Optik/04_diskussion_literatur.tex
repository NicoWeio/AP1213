\clearpage
\section{Diskussion}
	\label{sec:diskussion}
	Es f"allt auf, dass alle Methoden zur Bestimmung der Brennweite der bekannten Linse um etwa $\SI{10}{\percent}$ von den Herstellerangaben abweichen.
	Unter der Annahme, dass die Herstellerangaben korrekt sind, l"asst sich diese Abweichung durch eine ungenaue Bestimmung der Bildsch"arfe erkl"aren.
	Hierduch entstand bei der Messung der gr"o"ste Fehler.
	Dennoch sind die Werte insgesamt akzeptabel und best"atigen somit die Messmethoden zur Bestimmung der Brennweite.

	Die Messung zur chromatischen Abberation zeigt selbst unter Ber"ucksichtigung der Fehlertoleranzen einen Unterschied der Brennweiten bei rotem und blauem Licht.
	Der Effekt kann also bei empfindlichen Systemen, wie zum Beispiel Fotokameras nicht vernachl"assigt werden.

	\begin{thebibliography}{9}
	\label{sec:literaturverzeichnis}
		\bibitem{anleitung} Physikalisches Anf"angerpraktikum der TU Dortmund: Versuch Nr. 408 - Geometrische Optik. Stand: Januar 2013.
	\end{thebibliography}
