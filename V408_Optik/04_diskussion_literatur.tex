\clearpage
\section{Diskussion}
	\label{sec:diskussion}
	Alle Methoden zur Bestimmung der Brennweite liefern etwa einen Wert von $f = \SI{163}{\milli \meter}$. Dieser Weicht jedoch um $\SI{9}{\percent}$ von den Herstellerangaben ($f_\mathrm{Her} = \SI{150}{\milli \meter}$) ab.
	Unter der Annahme, dass die Herstellerangaben korrekt sind, legt das Ergebnis nahe, dass ein Systematischer Fehler vorliegt.
	Eventuell wurde der Sch"arfepunkt bei jeder Messung falsch gew"ahlt.
	Dennoch sind die Werte insgesamt akzeptabel und best"atigen somit die Messmethoden zur Bestimmung der Brennweite.

	Die Messung zur chromatischen Abberation zeigt selbst unter Ber"ucksichtigung der Fehlertoleranzen einen Unterschied der Brennweiten bei rotem und blauem Licht.
	Der Effekt kann also bei empfindlichen Systemen, wie zum Beispiel Fotokameras nicht vernachl"assigt werden.

	\begin{thebibliography}{9}
	\label{sec:literaturverzeichnis}
		\bibitem{anleitung} Physikalisches Anf"angerpraktikum der TU Dortmund: Versuch Nr. 408 - Geometrische Optik. Stand: Januar 2013.
	\end{thebibliography}
