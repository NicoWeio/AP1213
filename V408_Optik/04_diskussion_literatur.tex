\clearpage
\section{Diskussion}
	\label{sec:diskussion}
	Alle Methoden zur Bestimmung der Brennweite liefern etwa einen Wert von $f \approx \SI{163}{\milli \meter}$. Dieser Wert weicht jedoch um fast $\SI{9}{\percent}$ von den Herstellerangaben ($f_\mathrm{Her} = \SI{150}{\milli \meter}$) ab.
	Unter der Annahme, dass die Herstellerangaben korrekt sind, legt das Ergebnis nahe, dass ein Systematischer Fehler vorliegt.
	Eventuell wurde der Sch"arfepunkt bei jeder Messung falsch gew"ahlt.
	Dennoch sind die Werte insgesamt akzeptabel und best"atigen somit die Messmethoden zur Bestimmung der Brennweite und damit die Linsengleichung \eqref{linsengleichung}.
	Der Mittelwert der Abbildungsma"sstabverh"altnisse weicht mit $\SI{.97}{}$ zudem nur um $\SI{3}{\percent}$ vom Theoriewert $(B/G)/(b/g) = 1$ ab, wodurch Gleichung \eqref{abbildungsmassstab} ebenfalls best"atigt werden kann.

	Die Messung zur chromatischen Abberation zeigt selbst unter Ber"ucksichtigung der Fehlertoleranzen einen minimalen Unterschied der Brennweiten von $\Delta f = \SI{.78}{\milli \meter}$ bei rotem und blauem Licht.
	Der Effekt kann also bei empfindlichen Systemen, wie zum Beispiel Fotokameras nicht vernachl"assigt werden.

	\begin{thebibliography}{9}
	\label{sec:literaturverzeichnis}
		\bibitem{anleitung} Physikalisches Anf"angerpraktikum der TU Dortmund: Versuch Nr. 408 - Geometrische Optik. Stand: Januar 2013.
	\end{thebibliography}
