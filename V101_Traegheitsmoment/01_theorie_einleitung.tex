\section{Einleitung} % (fold)
\label{sec:einleitung}

	Bei diesem Versuch soll das Tr"agheitsmoment verschiedener K"orper gemessen werden. Dabei wird der Steiner'sche Satz verifiziert.

\section{Theorie} % (fold)
\label{sec:theorie}

	Rotationsbewegungen werden durch das Drehmoment $M$, die Winkelbeschleunigung $\dot\omega$ und das Tr"ageitsmoment $I$ charakterisiert.
	Das Gesamttr"agheitsmoment einer punktf"ormigen Masse $m$ im Abstand $r$ von der Drehachse ist gegeben durch $I = mr^2$.
	F"ur einen ausgedehnten K"orper gilt dementsprechend:

	\begin{equation}
		I = \sum_\mathrm{i} r_\mathrm{i}^2 \cdot m_\mathrm{i} \qquad .
	\end{equation}

	Bei infinitisimalen Massen ergibt sich f"ur das Gesamttr"agheitsmoment:

	\begin{equation}
		I = \int r^2 \mathrm{dm} \qquad .
	\end{equation}

	F"allt die Drehachse nicht mit einer Haupttr"agheitsachse zusammen, sondern ist um den Abstand $a$ verschoben, so gilt der Steiner'sche Satz:

	\begin{equation}
		I = I_\mathrm{s} + m \cdot a^2 \qquad .
	\end{equation}

	$I_\mathrm{s}$ stellt dabei das Tr"agheitsmoment bzgl. einer Drehachse, welche durch den Schwerpunkt des K"orpers geht w"urde, w"ahrend $m$ der Gesamtmasse des K"orpers entspricht.

	Wirkt bei einem drehbaren K"orper die Kraft $\vec{F}$ im Abstand $\vec{r}$ von der Drehachse, so wirkt auf den K"orper das Drehmoment $\vec{M} = \vec{F} \times \vec{r}$.

	Wird ein K"orper auf eine Drillachse gespannt und aus seiner Ruhelage um den Winkel $\varphi$ ausgelenkt, so wirkt auf ihn ein r"ucktreibendes Drehmoment durch eine Feder.

	Beim Loslassen f"uhrt der ausgelenkte K"orper eine harmonische Schwingung aus, wobei f"ur die Schwingungsdauer $T$ gilt:

	\begin{equation}
		T = 2\pi \sqrt{\frac{I}{D}} \qquad . \label{dauer}
	\end{equation}

	Dabei ist die Winkelrichtgr"o"se $D$ mit dem Drehmoment "uber Gleichung \eqref{D} verbunden:

	\begin{equation}
		M = D \varphi \qquad . \label{D}
	\end{equation}

	Bei Drehschwingungen verh"alt sich das System f"ur kleine Winkel $\varphi$ harmonisch.
	Die d"ampfende Wirkung ist bei diesem Versuch jedoch bei kleinen Winkeln sehr gro"s, sodass ein Winkel zwischen $\SI{40}{^\circ}$ und $\SI{60}{^\circ}$ gew"ahlt wird. 
	Die Winkelrichtgr"o"se $D$ kann statisch aus \eqref{D} durch Messen der Auslenkung $\varphi$ als Funktion der Kraft $F$ bestimmt werden oder dynamisch aus \eqref{dauer} durch Messung der Schwingungsdauer bei bekanntem Tr"agheitsmoment $I$ des K"orpers.