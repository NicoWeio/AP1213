\section{Diskussion}
\label{diskussion}
Es ist auff"allig, dass alle experimentellen Werte etwa $\SI{25}{\percent}$ unter den theoretischen Werten liegen.
Lediglich bei Stellung 1 (siehe Abb. \ref{fig:puppe1}) weicht der experimentelle Wert nur etwa $\SI{7}{\percent}$ vom Theoriewert ab.

Der gr"o"ste Fehler wird hier bei dem D"ampfungseffekt der Apparatur vermutet.
W"ahrend der Messung ist aufgefallen, dass die Schwingung bei mehr als 3 Perioden schon stark ged"ampft wurde.
K"urzere Schwingphasen konnten jedoch wegen der Reaktionszeit kaum gemessen werden.
Weil die Puppe in Stellung 1 mit der h"ochsten Frequenz geschwungen ist, liegt hier m"oglicherweise der kleinste Fehler vor.


Eine weitere Fehlerquelle war die Bestimmung der Winkelrichtgr"o"se $D$ mit Hilfe eines Kraftmessers.
Per Hand war eine optimale Ausrichtung des Kraftmessers nicht m"oglich.
Zudem sorgten innere Reibungskr"afte f"ur sprunghafte Wert"anderungen, was die Genauigkeit des Ger"ates in Frage stellt.
	

\begin{thebibliography}{9}
	\bibitem{anleitung} Physikalisches Anf"angerpraktikum der TU Dortmund: Versuch V101 - Das Tr"agheitsmoment. Stand: April 2013.
\end{thebibliography}
