\section{Diskussion}
\label{diskussion}
Es ist auff"allig, dass die experimentellen Werte f"ur die Holzkugel und den Zylinder etwa $\SI{100}{\percent}$ von den theoretischen Werten abweichen.
Lediglich bei Stellung 1 der Puppe (siehe Abb. \ref{fig:puppe1}) weicht der experimentelle Wert nur etwa $\SI{6}{\percent}$ vom Theoriewert ab.

Der gr"o"ste Fehler wird hier bei dem D"ampfungseffekt der Apparatur vermutet.
W"ahrend der Messung f"allt auf, dass die Schwingung bei mehr als 3 Perioden schon stark ged"ampft wurde.
K"urzere Schwingphasen konnten jedoch wegen der Reaktionszeit kaum gemessen werden.
Weil die Puppe in Stellung 1 mit der h"ochsten Frequenz geschwungen ist, liegt hier m"oglicherweise der kleinste Fehler vor.
Die D"ampfung schien bei langsameren Schwingungen einen gr"o"seren Effekt zu haben.

Eine weitere Fehlerquelle war die Bestimmung der Winkelrichtgr"o"se $D$ mit Hilfe eines Kraftmessers.
Innere Reibungskr"afte sorgten f"ur sprunghafte Wert"anderungen, was die Genauigkeit des Ger"ates in Frage stellt.

\begin{table}[h!]
	\begin{center}
		\caption{Relative Unterschiede zwischen theoretisch und experimentell ermittelten Werten f"ur das Tr"agheitsmoment $I$\label{tabelle:unterschiede}}
		\begin{tabular}{|c||c|}
			\hline
			K"orper & $\delta [\SI{}{\percent}]$ \\
			\hline 
			\hline
			Holzkugel & $\SI{96}{}$ \\
			Messingzylinder & $\SI{99}{}$ \\
			Puppe 1 & $\SI{6}{}$ \\
			Puppe 2 & $\SI{24}{}$ \\
			\hline 
		\end{tabular}
	\end{center}
\end{table}
	

\begin{thebibliography}{9}
	\bibitem{anleitung} Physikalisches Anf"angerpraktikum der TU Dortmund: Versuch V101 - Das Tr"agheitsmoment. \url{http://129.217.224.2/HOMEPAGE/PHYSIKER/BACHELOR/AP/SKRIPT/Traegheit.pdf}. Stand: April 2013.
\end{thebibliography}
